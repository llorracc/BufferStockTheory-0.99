\documentclass[BufferStockTheory]{subfiles}% LaTeX path to the root directory of the current project
\providecommand{\econtexRoot}{}
\renewcommand{\econtexRoot}{..}
\providecommand{\econtexPaths}{LaTeX}\renewcommand{\econtexPaths}{\econtexRoot/LaTeX/econtexPaths}

% WARNING: Different execution depending on whether
% 0. Being compiled as standalone document
% * Compile this file, then main, then this one again
% * Keep iterating until neither file changes
% 0. Being compiled as subfile of main document

%\onlyinsubfile{\externaldocument{BufferStockTheory}} % Get xrefs -- esp to appendix -- from main file; only works properly if main file has already been compiled; 

\begin{document}

\providecommand{\figName}{InequalityPFGICFHWCRIC}
\hypertarget{InequalityPFGICFHWCRIC}{}
\tikzset{node distance=6cm, auto, >=latex}
\begin{figure}[h]
  \label{fig:InequalityPFGICFHWCRIC}
  \centerline{
      \begin{tikzpicture}
        \node (thorn) {$\Pat$}; % ((\Rfree \DiscFac)^{1/\CRRA} \equiv )~~
\node (gamma) [right of = thorn] {$\PGro$};
\node (thorndef) [left of = thorn,xshift=4.8cm] {$((\Rfree \DiscFac)^{1/\CRRA} \equiv)$};
\node (rfree) [below of = thorn, xshift = 2.8cm, yshift = 2cm]{${\Rfree}$};
\node (ineq)  [below of = thorn, xshift = 2.8cm, yshift = 4.5cm]{$\neq$};
%\node (strix) [left of = gamma, xshift = 4.35cm, yshift = -0.6cm] {};
%\node (striy) [left of = gamma, xshift = 2.78cm, yshift = -0.9cm] {};
%\draw[-] (strix) to (striy); 
\draw[->] (thorn) to node{\PFGIC $(\equiv \Pat < \PGro)$} (gamma);
%\draw[->] (gamma) to [out = 200, in=-10] node[below]{\PFGIC} (thorn); %\st{\textrm{PF-GIC};
\draw[->] (thorn) to node [swap] {$(\Pat < \Rfree \equiv)~~\mathrm{\RIC}$} (rfree);
\draw[->] (gamma) to node {$\mathrm{\FHWC}~~(\equiv \PGro < \Rfree) $} (rfree);

      \end{tikzpicture}
  }
  \caption{Relation of {\RIC}, {\PFGIC}, and {\FHWC} in Perfect Foresight Model}
  \footnotesize{Arrows reflect the direction of the relationship; an arrowhead points to the larger of the two quantities being compared.  For example, the topmost arrow, pointing from $\Pat$ to $\PGro$, indicates that $\PGro > \Pat$.}
\end{figure}

\end{document}


\providecommand{\figName}{RelatePFGICFHWCRICPFFVAC} % Allows generic definition of hypertargets based on title of figure
\providecommand{\figFile}{\figName} %  and on filename
\hypertarget{\figFile}{}
\hypertarget{\figName}{}
\input{\FigDir/\figName} % Read in the tex to generate the figure


\providecommand{\figName}{Convergence-of-the-Consumption-Rules} % Allows generic definition of hypertargets based on title of figure
\providecommand{\figName}{cFuncsConverge} %  and on filename
\providecommand{\figFile}{cFuncsConverge} %  and on filename
\hypertarget{\figFile}{}
\hypertarget{\figName}{}
\input{\FigDir/\figName} % Read in the tex to generate the figure

\renewcommand{\figName}{Inequalities} % Allows generic definition of hypertargets based on title of figure
\renewcommand{\figFile}{\figName} %  and on filename
\hypertarget{\figFile}{}
\hypertarget{\figName}{}
\input{\FigDir/\figName} % Read in the tex to generate the figure
