\documentclass[BufferStockTheory]{subfiles}% LaTeX path to the root directory of the current project, from the directory in which this file resides
% and path to econtexPaths which defines the rest of the paths like \FigDir
\providecommand{\econtexRoot}{}\renewcommand{\econtexRoot}{.}
\providecommand{\econtexPaths}{}\renewcommand{\econtexPaths}{\econtexRoot/Resources/econtexPaths}
% The \commands below are required to allow sharing of the same base code via Github between TeXLive on a local machine and Overleaf (which is a proxy for "a standard distribution of LaTeX").  This is an ugly solution to the requirement that custom LaTeX packages be accessible, and that Overleaf seems to ignore symbolic links (even if they are relative links to valid locations)
\providecommand{\econtex}{\econtexRoot/Resources/texmf-local/tex/latex/econtex}
\providecommand{\econtexSetup}{\econtexRoot/Resources/texmf-local/tex/latex/econtexSetup}
\providecommand{\econtexShortcuts}{\econtexRoot/Resources/texmf-local/tex/latex/econtexShortcuts}
\providecommand{\econtexBibMake}{\econtexRoot/Resources/texmf-local/tex/latex/econtexBibMake}
\providecommand{\econtexBibStyle}{\econtexRoot/Resources/texmf-local/bibtex/bst/econtex}
\providecommand{\econtexBib}{economics}
\providecommand{\notes}{\econtexRoot/Resources/texmf-local/tex/latex/handout}
\providecommand{\handoutSetup}{\econtexRoot/Resources/texmf-local/tex/latex/handoutSetup}
\providecommand{\handoutShortcuts}{\econtexRoot/Resources/texmf-local/tex/latex/handoutShortcuts}
\providecommand{\handoutBibMake}{\econtexRoot/Resources/texmf-local/tex/latex/handoutBibMake}
\providecommand{\handoutBibStyle}{\econtexRoot/Resources/texmf-local/bibtex/bst/handout}

\providecommand{\FigDir}{\econtexRoot/Figures}
\providecommand{\CodeDir}{\econtexRoot/Code}
\providecommand{\DataDir}{\econtexRoot/Data}
\providecommand{\SlideDir}{\econtexRoot/Slides}
\providecommand{\TableDir}{\econtexRoot/Tables}
\providecommand{\ApndxDir}{\econtexRoot/Appendices}

\providecommand{\ResourcesDir}{\econtexRoot/Resources}
\ifnum\pdfshellescape=1
\providecommand{\rootFromOut}{..} % Path back to root directory from output-directory
\providecommand{\LaTeXGenerated}{\econtexRoot/LaTeX} % Put generated files in subdirectory
\providecommand{\EqDir}{\econtexRoot/Equations} % Put generated files in subdirectory
\else
\providecommand{\rootFromOut}{.} % Path back to root directory 
\providecommand{\LaTeXGenerated}{\econtexRoot/} % Put generated files in main directory (because not allowed in subdirectory)
\providecommand{\EqDir}{\econtexRoot/} % Put generated files in main directory
\fi
\providecommand{\econtexPaths}{\econtexRoot/Resources/econtexPaths}
\providecommand{\LaTeXInputs}{\econtexRoot/Resources/LaTeXInputs}


% WARNING: Different execution depending on whether
% 0. Being compiled as standalone document
% * Compile this file, then main, then this one again
% * Keep iterating until neither file changes
% 0. Being compiled as subfile of main document

%\onlyinsubfile{\externaldocument{BufferStockTheory}} % Get xrefs -- esp to appendix -- from main file; only works properly if main file has already been compiled; 

\begin{document}

\providecommand{\figName}{InequalityPFGICFHWCRIC}
\hypertarget{InequalityPFGICFHWCRIC}{}
\tikzset{node distance=6cm, auto, >=latex}
\begin{figure}[h]
  \label{fig:InequalityPFGICFHWCRIC}
  \centerline{
      \begin{tikzpicture}
        \node (thorn) {$\Pat$}; % ((\Rfree \DiscFac)^{1/\CRRA} \equiv )~~
\node (gamma) [right of = thorn] {$\PGro$};
\node (thorndef) [left of = thorn,xshift=4.8cm] {$((\Rfree \DiscFac)^{1/\CRRA} \equiv)$};
\node (rfree) [below of = thorn, xshift = 2.8cm, yshift = 2cm]{${\Rfree}$};
\node (ineq)  [below of = thorn, xshift = 2.8cm, yshift = 4.5cm]{$\neq$};
%\node (strix) [left of = gamma, xshift = 4.35cm, yshift = -0.6cm] {};
%\node (striy) [left of = gamma, xshift = 2.78cm, yshift = -0.9cm] {};
%\draw[-] (strix) to (striy); 
\draw[->] (thorn) to node{\PFGIC $(\equiv \Pat < \PGro)$} (gamma);
%\draw[->] (gamma) to [out = 200, in=-10] node[below]{\PFGIC} (thorn); %\st{\textrm{PF-GIC};
\draw[->] (thorn) to node [swap] {$(\Pat < \Rfree \equiv)~~\mathrm{\RIC}$} (rfree);
\draw[->] (gamma) to node {$\mathrm{\FHWC}~~(\equiv \PGro < \Rfree) $} (rfree);

      \end{tikzpicture}
  }
  \caption{Relation of {\RIC}, {\PFGIC}, and {\FHWC} in Perfect Foresight Model}
  \footnotesize{Arrows reflect the direction of the relationship; an arrowhead points to the larger of the two quantities being compared.  For example, the topmost arrow, pointing from $\Pat$ to $\PGro$, indicates that $\PGro > \Pat$.}
\end{figure}

\end{document}


\providecommand{\figName}{RelatePFGICFHWCRICPFFVAC} % Allows generic definition of hypertargets based on title of figure
\providecommand{\figFile}{\figName} %  and on filename
\hypertarget{\figFile}{}
\hypertarget{\figName}{}
\input{\FigDir/\figName} % Read in the tex to generate the figure


\providecommand{\figName}{Convergence-of-the-Consumption-Rules} % Allows generic definition of hypertargets based on title of figure
\providecommand{\figName}{cFuncsConverge} %  and on filename
\providecommand{\figFile}{cFuncsConverge} %  and on filename
\hypertarget{\figFile}{}
\hypertarget{\figName}{}
\input{\FigDir/\figName} % Read in the tex to generate the figure

\renewcommand{\figName}{Inequalities} % Allows generic definition of hypertargets based on title of figure
\renewcommand{\figFile}{\figName} %  and on filename
\hypertarget{\figFile}{}
\hypertarget{\figName}{}
\input{\FigDir/\figName} % Read in the tex to generate the figure
