% LaTeX path to the root directory of the current project, from the directory in which this file resides
% and path to econtexPaths which defines the rest of the paths like \FigDir
\providecommand{\econtexRoot}{}\renewcommand{\econtexRoot}{.}
\providecommand{\econtexPaths}{}\renewcommand{\econtexPaths}{\econtexRoot/Resources/econtexPaths}
% The \commands below are required to allow sharing of the same base code via Github between TeXLive on a local machine and Overleaf (which is a proxy for "a standard distribution of LaTeX").  This is an ugly solution to the requirement that custom LaTeX packages be accessible, and that Overleaf seems to ignore symbolic links (even if they are relative links to valid locations)
\providecommand{\econtex}{\econtexRoot/Resources/texmf-local/tex/latex/econtex}
\providecommand{\econtexSetup}{\econtexRoot/Resources/texmf-local/tex/latex/econtexSetup}
\providecommand{\econtexShortcuts}{\econtexRoot/Resources/texmf-local/tex/latex/econtexShortcuts}
\providecommand{\econtexBibMake}{\econtexRoot/Resources/texmf-local/tex/latex/econtexBibMake}
\providecommand{\econtexBibStyle}{\econtexRoot/Resources/texmf-local/bibtex/bst/econtex}
\providecommand{\econtexBib}{economics}
\providecommand{\notes}{\econtexRoot/Resources/texmf-local/tex/latex/handout}
\providecommand{\handoutSetup}{\econtexRoot/Resources/texmf-local/tex/latex/handoutSetup}
\providecommand{\handoutShortcuts}{\econtexRoot/Resources/texmf-local/tex/latex/handoutShortcuts}
\providecommand{\handoutBibMake}{\econtexRoot/Resources/texmf-local/tex/latex/handoutBibMake}
\providecommand{\handoutBibStyle}{\econtexRoot/Resources/texmf-local/bibtex/bst/handout}

\providecommand{\FigDir}{\econtexRoot/Figures}
\providecommand{\CodeDir}{\econtexRoot/Code}
\providecommand{\DataDir}{\econtexRoot/Data}
\providecommand{\SlideDir}{\econtexRoot/Slides}
\providecommand{\TableDir}{\econtexRoot/Tables}
\providecommand{\ApndxDir}{\econtexRoot/Appendices}

\providecommand{\ResourcesDir}{\econtexRoot/Resources}
\ifnum\pdfshellescape=1
\providecommand{\rootFromOut}{..} % Path back to root directory from output-directory
\providecommand{\LaTeXGenerated}{\econtexRoot/LaTeX} % Put generated files in subdirectory
\providecommand{\EqDir}{\econtexRoot/Equations} % Put generated files in subdirectory
\else
\providecommand{\rootFromOut}{.} % Path back to root directory 
\providecommand{\LaTeXGenerated}{\econtexRoot/} % Put generated files in main directory (because not allowed in subdirectory)
\providecommand{\EqDir}{\econtexRoot/} % Put generated files in main directory
\fi
\providecommand{\econtexPaths}{\econtexRoot/Resources/econtexPaths}
\providecommand{\LaTeXInputs}{\econtexRoot/Resources/LaTeXInputs}

\documentclass[\econtexRoot/BufferStockTheory]{subfiles}
% LaTeX path to the root directory of the current project, from the directory in which this file resides
% and path to econtexPaths which defines the rest of the paths like \FigDir
\providecommand{\econtexRoot}{}\renewcommand{\econtexRoot}{.}
\providecommand{\econtexPaths}{}\renewcommand{\econtexPaths}{\econtexRoot/Resources/econtexPaths}
% The \commands below are required to allow sharing of the same base code via Github between TeXLive on a local machine and Overleaf (which is a proxy for "a standard distribution of LaTeX").  This is an ugly solution to the requirement that custom LaTeX packages be accessible, and that Overleaf seems to ignore symbolic links (even if they are relative links to valid locations)
\providecommand{\econtex}{\econtexRoot/Resources/texmf-local/tex/latex/econtex}
\providecommand{\econtexSetup}{\econtexRoot/Resources/texmf-local/tex/latex/econtexSetup}
\providecommand{\econtexShortcuts}{\econtexRoot/Resources/texmf-local/tex/latex/econtexShortcuts}
\providecommand{\econtexBibMake}{\econtexRoot/Resources/texmf-local/tex/latex/econtexBibMake}
\providecommand{\econtexBibStyle}{\econtexRoot/Resources/texmf-local/bibtex/bst/econtex}
\providecommand{\econtexBib}{economics}
\providecommand{\notes}{\econtexRoot/Resources/texmf-local/tex/latex/handout}
\providecommand{\handoutSetup}{\econtexRoot/Resources/texmf-local/tex/latex/handoutSetup}
\providecommand{\handoutShortcuts}{\econtexRoot/Resources/texmf-local/tex/latex/handoutShortcuts}
\providecommand{\handoutBibMake}{\econtexRoot/Resources/texmf-local/tex/latex/handoutBibMake}
\providecommand{\handoutBibStyle}{\econtexRoot/Resources/texmf-local/bibtex/bst/handout}

\providecommand{\FigDir}{\econtexRoot/Figures}
\providecommand{\CodeDir}{\econtexRoot/Code}
\providecommand{\DataDir}{\econtexRoot/Data}
\providecommand{\SlideDir}{\econtexRoot/Slides}
\providecommand{\TableDir}{\econtexRoot/Tables}
\providecommand{\ApndxDir}{\econtexRoot/Appendices}

\providecommand{\ResourcesDir}{\econtexRoot/Resources}
\ifnum\pdfshellescape=1
\providecommand{\rootFromOut}{..} % Path back to root directory from output-directory
\providecommand{\LaTeXGenerated}{\econtexRoot/LaTeX} % Put generated files in subdirectory
\providecommand{\EqDir}{\econtexRoot/Equations} % Put generated files in subdirectory
\else
\providecommand{\rootFromOut}{.} % Path back to root directory 
\providecommand{\LaTeXGenerated}{\econtexRoot/} % Put generated files in main directory (because not allowed in subdirectory)
\providecommand{\EqDir}{\econtexRoot/} % Put generated files in main directory
\fi
\providecommand{\econtexPaths}{\econtexRoot/Resources/econtexPaths}
\providecommand{\LaTeXInputs}{\econtexRoot/Resources/LaTeXInputs}

\usepackage{\LaTeXInputs/econtex_onlyinsubfile} 

\onlyinsubfile{\externaldocument{\LaTeXGenerated/BufferStockTheory}} % Get xrefs -- esp to appendix -- from main file; only works properly if main file has already been compiled;
\begin{document}

\section{Existence of a Concave Consumption Function}\label{sec:ApndxcExists}

To show that \eqref{\localorexternallabel{eq:veqn}} defines a sequence of continuously differentiable strictly increasing concave functions $\{\cFunc_{T},\cFunc_{T-1},...,\cFunc_{T-k}\}$, we start with a definition.  We will say that a function $\nFunc(z)$ is `nice' if it satisfies
\begin{quote}
\begin{enumerate}\setlength{\itemsep}{0.0ex}
\item $\nFunc(z)$ is well-defined iff $z>0$

\item $\nFunc(z)$ is strictly increasing

\item $\nFunc(z)$ is strictly concave

\item $\nFunc(z)$ is $ \mathbf{C}^{3}$

\item $\nFunc(z)<0$

\item $\lim_{z\downarrow 0}~\nFunc(z) =-\infty $.

\end{enumerate}
\end{quote}


(Notice that an implication of niceness is that $\lim_{z \downarrow 0} \nFunc^{\prime}(z) = \infty.$)

Assume that some $\vFunc_{t+1}$ is nice.  Our objective is to show that this
implies $\vFunc_{t}$ is also nice; this is sufficient to establish that
$\vFunc_{t-n}$ is nice by induction for all $n > 0$ because $\vFunc_{T}(\mRat)
=\uFunc(\mRat) $ and $\uFunc(\mRat)=\mRat^{1-\CRRA}/(1-\CRRA)$ is nice by inspection.

Now define an end-of-period value function $\mathfrak{v}_{t}(a) $ as
\begin{equation}
\mathfrak{v}_{t}(a) =\DiscFac \Ex_{t}\left[{\PGro}_{t+1}^{1-\CRRA}\vFunc_{t+1}( {\mathcal{R}}_{t+1} a+{\tShkAll}_{t+1}) \right]. \label{eq:vEnd}
\end{equation}

Since there is a positive probability that $\tShkAll_{t+1}$ will
attain its minimum of zero and since $\mathcal{R}_{t+1}>0$, it
is clear that $\lim_{\aRat \downarrow 0} \mathfrak{v}_{t}(a) = -\infty$
and $\lim_{\aRat \downarrow 0} \mathfrak{v}^{\prime}_{t}(a) = \infty$.  So
$\mathfrak{v}_{t}(a) $ is well-defined iff $\aRat>0$; it is similarly
straightforward to show the other properties required for $\mathfrak{v}_{t}(a) $ to
be nice.  (See Hiraguchi~\citeyearpar{hiraguchiBSProofs}.)

Next define $\underline{\vFunc}_{t}(\mRat,\cRat)$ as
\begin{equation}
\underline{\vFunc}_{t}(\mRat,\cRat)=\uFunc(c)+\mathfrak{v}_{t}(\mRat-c)
\end{equation}
which is $\mathbf{C}^{3}$ since $\mathfrak{v}_{t}$ and $\uFunc$ are both
$\mathbf{C}^{3},$ and note that our problem's value function defined
in \eqref{\localorexternallabel{eq:veqn}} can be written as
\begin{equation}\begin{gathered}\begin{aligned}
\vFunc_{t}(\mRat)  & =  \max_{\cRat}~\underline{\vFunc}_{t}(\mRat,\cRat).
\end{aligned}\end{gathered}\end{equation}

$\underline{\vFunc}_{t}$ is well-defined if and only if $0<c<m$.  Furthermore,
$\lim_{c \downarrow
  0}\underline{\vFunc}_{t}(\mRat,\cRat)=\lim_{c\uparrow m} \underline{\vFunc}_{t}(\mRat,\cRat)=-\infty $, $\frac{\partial ^{2}\underline{\vFunc}_{t}(\mRat,\cRat)}{\partial c^{2}}%
<0$, $\lim_{c \downarrow 0}\frac{\partial \underline{\vFunc}_{t}(\mRat,\cRat)}{\partial c}%
=+\infty $, and $\lim_{c\uparrow m} \frac{\partial \underline{\vFunc}_{t}(\mRat,\cRat)}{%
\partial c}=-\infty $. It follows that the $\cFunc_{t}(\mRat)$ defined by
\begin{equation}\begin{gathered}\begin{aligned}
\cFunc_{t}(\mRat)  & = \underset{0<c<m}{\arg \max }~\underline{\vFunc}_{t}(\mRat,\cRat)
\end{aligned}\end{gathered}\end{equation}
exists and is unique, and \eqref{\localorexternallabel{eq:veqn}} has an internal
solution that satisfies
\begin{equation}
\uFunc^{\prime }(\cFunc_{t}(\mRat))=\mathfrak{v}_{t}^{\prime }(\mRat-\cFunc_{t}(\mRat))  \label{eq:consumptionf}.
\end{equation}


Since both $\uFunc$ and $\mathfrak{v}_{t}$ are strictly concave, both
$\cFunc_{t}(\mRat)$ and $\aFunc_{t}(\mRat)=\mRat-\cFunc_{t}(\mRat)$
are strictly increasing. Since both $\uFunc$ and $\mathfrak{v}_{t}$ are
three times continuously differentiable, using \eqref{\localorexternallabel{eq:consumptionf}} we can conclude that
$\cFunc_{t}(\mRat)$ is continuously differentiable and
\begin{equation}\begin{gathered}\begin{aligned}
\cFunc_{t}^{\prime }(\mRat)  & = \frac{\mathfrak{v}_{t}^{\prime \prime }({\aFunc}_{t}(\mRat))  }{\uFunc^{\prime \prime }(\cFunc_{t}(\mRat))+\mathfrak{v}_{t}^{\prime \prime }({\aFunc}_{t}(\mRat))}.
\end{aligned}\end{gathered}\end{equation}

Similarly we can easily show that $\cFunc_{t}(\mRat)$ is twice
continuously differentiable (as is $\aFunc_{t}(\mRat)$) (See
Appendix~\ref{sec:CIsTwiceDifferentiable}.)  This implies that
$\vFunc_{t}(\mRat)$ is nice, since
$\vFunc_{t}(\mRat)=\uFunc(\cFunc_{t}(\mRat))+\mathfrak{v}_{t}({\aFunc}_{t}(\mRat))$.

\hypertarget{cFunc-is-Twice-Continuously-Differentiable}{}
\section{$\cFunc_{t}(\mRat)$ is Twice Continuously Differentiable}\label{sec:CIsTwiceDifferentiable}

First we show that $\cFunc_{t}(\mRat)$ is $\mathbf{C}^{1}.$ Define $y$ as
$y\equiv m+dm$.
  Since $\uFunc^{\prime }\left( \cFunc_{t}(y)\right) -\uFunc^{\prime }\left(
    \cFunc_{t}(\mRat)\right) =\mathfrak{v}_{t}^{\prime
  }({\aFunc}_{t}(y))-\mathfrak{v}_{t}^{\prime }({\aFunc}_{t}(\mRat))$ and $
  \frac{{\aFunc}_{t}(y)-{\aFunc}_{t}(\mRat)}{dm}=1-\frac{\cFunc_{t}(y)-\cFunc_{t}(\mRat)}{dm},$
  
\begin{align*}
%  \lefteqn{
  \frac{\mathfrak{v}_{t}^{\prime }({\aFunc}_{t}(y))-\mathfrak{v}_{t}^{\prime }({\aFunc}_{t}(\mRat))}{{\aFunc}_{t}(y)-{\aFunc}_{t}(\mRat)} %  }
  & =   
       \left( \frac{\uFunc^{\prime }\left( \cFunc_{t}(y)\right) -\uFunc^{\prime }\left( \cFunc_{t}(\mRat)\right) }{\cFunc_{t}(y)-\cFunc_{t}(\mRat)}+\frac{\mathfrak{v}_{t}^{\prime }({\aFunc}_{t}(y))-\mathfrak{v}_{t}^{\prime }({\aFunc}_{t}(\mRat))}{{\aFunc}_{t}(y)-{\aFunc}_{t}(\mRat)}\right) \frac{\cFunc_{t}(y)-\cFunc_{t}(\mRat)}{dm}
\end{align*}
Since $\cFunc_{t}$ and $\aFunc_{t}$ are continuous and increasing, $\underset{
dm\rightarrow +0}{\lim }\frac{\uFunc^{\prime }\left( \cFunc_{t}(y)\right) -\uFunc^{\prime
}\left( \cFunc_{t}(\mRat)\right) }{\cFunc_{t}(y)-\cFunc_{t}(\mRat)}<0$ and
$\underset{dm\rightarrow+0}{\lim }\frac{\mathfrak{v}_{t}^{\prime }({\aFunc}_{t}(y))-\mathfrak{v}_{t}^{\prime }({\aFunc}_{t}(\mRat))}{
{\aFunc}_{t}(y)-{\aFunc}_{t}(\mRat)}< 0$
are satisfied. Then $\frac{\uFunc^{\prime }\left(
\cFunc_{t}(y)\right) -\uFunc^{\prime }\left( \cFunc_{t}(\mRat)\right) }{\cFunc_{t}(y)-\cFunc_{t}(\mRat)}+
\frac{\mathfrak{v}_{t}^{\prime }({\aFunc}_{t}(y))-\mathfrak{v}_{t}^{\prime }({\aFunc}_{t}(\mRat))}{{\aFunc}_{t}(y)-{\aFunc}_{t}(\mRat)}
<0$ for sufficiently small $dm$.
 Hence we obtain a well-defined equation:

\begin{equation*}
\frac{\cFunc_{t}(y)-\cFunc_{t}(\mRat)}{dm}=\frac{\frac{\mathfrak{v}_{t}^{\prime
}({\aFunc}_{t}(y))-\mathfrak{v}_{t}^{\prime }({\aFunc}_{t}(\mRat))}{{\aFunc}_{t}(y)-{\aFunc}_{t}(\mRat)}}{\frac{\uFunc^{\prime
}\left( \cFunc_{t}(y)\right) -\uFunc^{\prime }\left( \cFunc_{t}(\mRat)\right) }{
\cFunc_{t}(y)-\cFunc_{t}(\mRat)}+\frac{\mathfrak{v}_{t}^{\prime }({\aFunc}_{t}(y))-\mathfrak{v}_{t}^{\prime }({\aFunc}_{t}(\mRat))
}{{\aFunc}_{t}(y)-{\aFunc}_{t}(\mRat)}}.
\end{equation*}
This implies that the right-derivative, $\cFunc_{t}^{\prime +}(\mRat)$ is
well-defined and

\begin{equation*}
\cFunc_{t}^{\prime +}(\mRat)=\frac{\mathfrak{v}_{t}^{\prime \prime }({\aFunc}_{t}(\mRat))}{\uFunc^{\prime \prime
}(\cFunc_{t}(\mRat))+\mathfrak{v}_{t}^{\prime \prime }({\aFunc}_{t}(\mRat))}.
\end{equation*}

Similarly we can show that $\cFunc_{t}^{\prime +}(\mRat)=\cFunc_{t}^{\prime -}(\mRat)$,
which means $\cFunc_{t}^{\prime }(\mRat)$ exists. Since $\mathfrak{v}_{t}$ is
$\mathbf{C}^{3}$, $ \cFunc_{t}^{\prime }(\mRat)$ exists and is continuous.
$\cFunc_{t}^{\prime }(\mRat)$ is differentiable because
$\mathfrak{v}_{t}^{\prime \prime }$ is $\mathbf{C}^{1}$, $ \cFunc_{t}(\mRat)$
is $\mathbf{C}^{1}$ and $\uFunc^{\prime \prime
}(\cFunc_{t}(\mRat))+\mathfrak{v}_{t}^{\prime \prime }\left( {\aFunc}_{t}(\mRat)\right)
<0$. $\cFunc_{t}^{\prime \prime }(\mRat)$ is given by
\begin{equation}
\cFunc_{t}^{\prime \prime }(\mRat)=\frac{{\aRat}_{t}^{\prime }(\mRat)\mathfrak{v}_{t}^{\prime \prime
\prime }({\aRat}_{t})\left[ \uFunc^{\prime \prime }(c_{t})+\mathfrak{v}_{t}^{\prime \prime }({\aRat}_{t})
\right] -\mathfrak{v}_{t}^{\prime \prime }({\aRat}_{t})\left[ c_{t}^{\prime }\uFunc^{\prime \prime
\prime }(c_{t})+{\aRat}_{t}^{\prime }\mathfrak{v}_{t}^{\prime \prime \prime }({\aRat}_{t})\right] }{
\left[ \uFunc^{\prime \prime }(c_{t})+\mathfrak{v}_{t}^{\prime \prime }({\aRat}_{t})\right] ^{2}}.
\end{equation}
Since $\mathfrak{v}_{t}^{\prime \prime }({\aFunc}_{t}(\mRat))$ is continuous,
$\cFunc_{t}^{\prime \prime }(\mRat)$ is also continuous.

\hypertarget{It-Is-A-Contraction-Mapping}{}
\section{Proof that $\TMap$ Is a Contraction Mapping}\label{sec:Tcomplete}

We must show that our operator $\TMap$ satisfies all of Boyd's
conditions.

Boyd's operator $\BoydT$ maps from
$\mathcal{C}_{\phiFunc}(\mathscr{A},\mathscr{B})$ to
$\mathcal{C}(\mathscr{A},\mathscr{B}).$ A preliminary requirement is
therefore that $\{\TMap{\zFunc}\}$ be continuous for any $\phiFunc-$bounded $\zFunc$,
$\{\TMap{\zFunc}\}\in~\mathcal{C}(\mathbb{R}_{++},\mathbb{R})$.  This is not
difficult to show; see Hiraguchi~\citeyearpar{hiraguchiBSProofs}.

Consider condition (1). For this problem,
\begin{align*}
\{\TMap{\mathfrak{\xFunc}}\}(\mRat_{t}) &\mbox{~is}\underset{\cRat_{t} \in
[\MinMinMPC \mRat_{t}, \MaxMPC \mRat_{t}]
}\max \left\{
\uFunc(c_{t})+\DiscFac \Ex_{t}\left[ {\PGro}_{t+1}^{1-\CRRA }{\mathfrak{\xFunc}}
\left( {\mRat}_{t+1}\right) \right] \right\}  \notag  \label{eq:condition1}
\\
\{\TMap{\mathfrak{\yFunc}}\}(\mRat_{t}) &\mbox{~is}\underset{\cRat_{t} \in
[\MinMinMPC \mRat_{t}, \MaxMPC \mRat_{t}]
}\max \left\{
\uFunc(c_{t})+\DiscFac \Ex_{t}\left[ {\PGro}_{t+1}^{1-\CRRA }{\mathfrak{\yFunc}}
\left( {\mRat}_{t+1}\right) \right] \right\} ,  \notag
\end{align*}%
so ${\mathfrak{\xFunc}}(\bullet) \leq {\mathfrak{\yFunc}}(\bullet)$ implies $\{\TMap{\mathfrak{\xFunc}}\}(\mRat_{t}) \leq \{\TMap{\mathfrak{\yFunc}} \}(\mRat_{t})$ by inspection.\footnote{For a fixed $\mRat_{t}$, recall that ${\mRat}_{t+1}$ is just a function of $c_{t}$ and the
stochastic shocks.}

Condition (2) requires that $\{\TMap\mathbf{0}\}\in \mathcal{C}_{\phiFunc}\left(\mathscr{A},\mathscr{B}\right)$. By definition,
\begin{equation*}
\{\TMap \mathbf{0}\}(\mRat_{t}) = \max_{\cRat_{t} \in
[\MinMinMPC \mRat_{t}, \MaxMPC \mRat_{t}]
}\left\{ \left( \frac{\cRat_{t}^{1-\CRRA }}{1-\CRRA }\right) +\DiscFac 0\right\}
\end{equation*}
the solution to which is patently
$\uFunc(\MaxMPC \mRat_{t})$. Thus, condition (2)
will hold if $(\MaxMPC \mRat_{t})^{1-\CRRA}$ is $\phiFunc$-bounded.  We use
the bounding function
\begin{equation}\begin{gathered}\begin{aligned}
  \label{eq:phiFunc}
  \phiFunc(\mRat)  & = \eta + \mRat^{1-\CRRA},
\end{aligned}\end{gathered}\end{equation}
for some real scalar $\eta > 0$ whose value will be determined in the
course of the proof. Under this definition of $\phiFunc$,
$\{\TMap\mathbf{0}\}(\mRat_{t})= \uFunc(\MaxMPC \mRat_{t})$
is clearly
$\phiFunc$-bounded.

Finally, we turn to condition (3), $\{\TMap({\zFunc} +\zeta\phiFunc
)\}(\mRat_{t}) \leq \{\TMap{\zFunc}\}(\mRat_{t}) +\zeta \Shrinker
\phiFunc(\mRat_{t}).$ The proof will be more compact if we define
$\breve{\cFunc}$ and $\breve{\aFunc}$ as the consumption and assets
functions\footnote{Section \ref{sec:cExists} proves existence of a
  continuously differentiable consumption function, which implies the
  existence of a corresponding continuously differentiable assets
  function.}  associated with $\TMap{\zFunc}$ and $\hat{\cFunc}$ and
$\hat{\aFunc}$ as the functions associated with $\TMap({\zFunc+\zeta
  \phiFunc})$; using this notation, condition (3) can be rewritten
\begin{align*}
\uFunc(\hat{\cFunc})+\DiscFac \{\EEndMap (\zFunc+\zeta \phiFunc) \}(\hat{\aFunc})  & \leq  \uFunc(\breve{\cFunc})+\DiscFac \{\EEndMap \zFunc \}(\breve{\aFunc})  + \zeta \Shrinker \phiFunc.
\end{align*}

Now note that if we force the $\smile$ consumer to consume the amount that is
optimal for the $\wedge$ consumer, value for the $\smile$ consumer must decline (at least weakly).  That is,
\begin{align*}
\uFunc(\hat{\cFunc})+\DiscFac \{\EEndMap \zFunc \}(\hat{\aFunc})  & \leq \uFunc(\breve{\cFunc})+\DiscFac \{\EEndMap \zFunc \}(\breve{\aFunc})
.
\end{align*}
Thus, condition (3) will certainly hold under the stronger condition
\begin{align*}
\uFunc(\hat{\cFunc})+\DiscFac\{\EEndMap (\zFunc+\zeta \phiFunc) \}(\hat{\aFunc})  & \leq  \uFunc(\hat{\cFunc})+\DiscFac\{\EEndMap \zFunc \}(\hat{\aFunc})  + \zeta \Shrinker \phiFunc
\\ \DiscFac\{\EEndMap (\zFunc+\zeta \phiFunc) \}(\hat{\aFunc})  & \leq  \DiscFac\{\EEndMap \zFunc  \}(\hat{\aFunc})  + \zeta \Shrinker \phiFunc
\\ \DiscFac\zeta \{\EEndMap \phiFunc \}(\hat{\aFunc})  & \leq  \zeta \Shrinker \phiFunc
\\ \DiscFac\{\EEndMap \phiFunc \}(\hat{\aFunc})  & \leq  \Shrinker \phiFunc
\\ \DiscFac\{\EEndMap \phiFunc \}(\hat{\aFunc})   & < \phiFunc %\label{eq:reqCondWeak}
.
\end{align*}
where the last line follows because $0 < \Shrinker < 1$ by assumption.\footnote{The remainder of the proof could be reformulated using the second-to-last line at a small cost to intuition.}

Using $\phiFunc(\mRat)= \eta + \mRat^{1-\CRRA}$
and defining $\hat{\aRat}_{t}=\hat{\aFunc}(\mRat_{t})$, this condition is
\begin{align*}
\DiscFac \Ex_{t}[{\PGro}_{t+1}^{1-\CRRA}(\hat{\aRat}_{t}\Rnorm_{t+1}+\tShkAll_{t+1})^{1-\CRRA}]-\mRat_{t}^{1-\CRRA}  & < \eta(1-\underbrace{\DiscFac\Ex_{t}{\PGro}_{t+1}^{1-\CRRA}}_{=\DiscAlt})
\end{align*}
which by imposing \PFFVAC~(equation \eqref{\localorexternallabel{eq:PFFVAC}}, which says $\DiscAlt<1$) can be rewritten as:
\begin{equation}\begin{gathered}\begin{aligned}
 \eta>\frac{\DiscFac \Ex_{t}\left[{\PGro}_{t+1}^{1-\CRRA}(\hat{\aRat}_{t}\Rnorm_{t+1}+\tShkAll_{t+1})^{1-\CRRA}\right]-\mRat_{t}^{1-\CRRA}}{1-\DiscAlt}\label{eq:KeyCondition}.
\end{aligned}\end{gathered}\end{equation}

But since $\eta$ is an arbitrary constant that we can pick, the proof thus reduces to showing that the numerator of \eqref{\localorexternallabel{eq:KeyCondition}} is bounded from above:
\begin{equation}\begin{gathered}\begin{aligned}%
  \lefteqn{\pNotZero\DiscFac\Ex_{t}\left[{\PGro}_{t+1}^{1-\CRRA}(\hat{\aRat}_{t}\Rnorm_{t+1}+\tShkEmp_{t+1}/\pNotZero)^{1-\CRRA}\right]
 +\pZero\DiscFac\Ex_{t}\left[{\PGro}_{t+1}^{1-\CRRA}(\hat{\aRat}_{t}\Rnorm_{t+1})^{1-\CRRA}\right]-\mRat_{t}^{1-\CRRA}\notag~~~}  \\ 
 ~~~\leq & \pNotZero\DiscFac\Ex_{t}\left[{\PGro}_{t+1}^{1-\CRRA}((1-\MaxMPC)\mRat_{t}\Rnorm_{t+1}+\tShkEmp_{t+1}/\pNotZero)^{1-\CRRA}\right]
 +\pZero\DiscFac\Rfree^{1-\CRRA}((1-\MaxMPC)\mRat_{t})^{1-\CRRA}-\mRat_{t}^{1-\CRRA}\notag\\
 ~~~= & \pNotZero\DiscFac\Ex_{t}\left[{\PGro}_{t+1}^{1-\CRRA}((1-\MaxMPC)\mRat_{t}\Rnorm_{t+1}+\tShkEmp_{t+1}/\pNotZero)^{1-\CRRA}\right]
 +\mRat_{t}^{1-\CRRA}\left(\pZero\DiscFac\Rfree^{1-\CRRA}\left(\pZero^{1/\CRRA}\frac{(\Rfree\DiscFac)^{1/\CRRA}}{\Rfree}\right)^{1-\CRRA}-1\right)\notag\\
 ~~~ =  & \pNotZero\DiscFac\Ex_{t}\left[{\PGro}_{t+1}^{1-\CRRA}((1-\MaxMPC)\mRat_{t}\Rnorm_{t+1}+\tShkEmp_{t+1}/\pNotZero)^{1-\CRRA}\right]
 +\mRat_{t}^{1-\CRRA}\left(\underbrace{\pZero^{1/\CRRA}\frac{(\Rfree\DiscFac)^{1/\CRRA}}{\Rfree}}_{<1~\text{by
       \WRIC}}-1\right) \label{eq:WRICBites} \\
 ~~~< &\pNotZero\DiscFac\Ex_{t}\left[{\PGro}_{t+1}^{1-\CRRA}(\underline{\tShkEmp}/\pNotZero)^{1-\CRRA}\right]=\DiscAlt\pNotZero^{\CRRA}\underline{\tShkEmp}^{1-\CRRA} \notag
 .
\end{aligned}\end{gathered}\end{equation}

We can thus conclude that equation \eqref{\localorexternallabel{eq:KeyCondition}} will certainly hold for any:
\begin{equation}\begin{gathered}\begin{aligned}
 \eta>\underline{\eta}=\frac{\DiscAlt\pNotZero^{\CRRA}\underline{\tShkEmp}^{1-\CRRA}}{1-\DiscAlt}
\end{aligned}\end{gathered}\end{equation}
which is a positive finite number under our assumptions.

The proof that $\TMap$ defines a contraction mapping under the
conditions \eqref{\localorexternallabel{eq:WRIC}} and \eqref{\localorexternallabel{eq:FVAC}} is
now complete.

\subsection{$\TMap$ and $\vFunc$}

In defining our operator $\TMap$ we made the restriction
$\MinMinMPC \mRat_{t} \leq c_{t} \leq \MaxMPC \mRat_{t}$.  However,
in the discussion of the consumption function bounds, we
showed only (in \eqref{\localorexternallabel{eq:cBounds}}) that $\MinMinMPC_{t} \mRat_{t} \leq \cFunc_{t}(\mRat_{t})
\leq \MaxMPC_{t} \mRat_{t}$.  (The difference is in the presence
or absence of time subscripts on the MPC's.)
  We have therefore
not proven (yet) that the sequence of value functions \eqref{\localorexternallabel{eq:veqn}} defines a contraction mapping.

Fortunately, the proof of that proposition is identical to the proof above, except that we must replace
$\MaxMPC$ with $\MaxMPC_{T-1}$ and the \WRIC~must be
replaced by a slightly stronger (but still quite weak) condition.  The place where these
conditions have force is in the step at \eqref{\localorexternallabel{eq:WRICBites}}.
Consideration of the prior two equations reveals that
a sufficient stronger condition is
\begin{align*}
    \pZero \DiscFac (\Rfree (1-\MaxMPC_{T-1}))^{1-\CRRA}  & < 1
\\  (\pZero \DiscFac)^{1/(1-\CRRA)}  (1-\MaxMPC_{T-1})  & > 1
\\  (\pZero \DiscFac)^{1/(1-\CRRA)}  (1-(1+\MinMPS)^{-1})  & > 1
\end{align*}
where we have used \eqref{\localorexternallabel{eq:MaxMPCInv}} for $\MaxMPC_{T-1}$ (and in the second step the reversal of the inequality occurs because we have assumed $\CRRA > 1$ so that we are exponentiating both sides by the negative number $1-\CRRA$).  To see that this is a weak condition, note that for small values of
$\pZero$ this expression can be further simplified using $(1+\MinMPS)^{-1}
\approx 1-\MinMPS$ so that it becomes
\begin{align*}
  (\pZero \DiscFac)^{1/(1-\CRRA)}  \MinMPS  & > 1
\\  (\pZero \DiscFac)  \pZero^{(1-\CRRA)/\CRRA} \PatR^{1-\CRRA}  & < 1
\\  \DiscFac  \pZero^{1/\CRRA} \PatR^{1-\CRRA}  & < 1.
\end{align*}

Calling the weak return patience factor $\PatR^{\wp}=\wp^{1/\CRRA}\PatR$ and
recalling that the \WRIC~was $\PatR^{\wp}<1$, the expression on the LHS
above is $\DiscFac \PatR^{-\CRRA}$ times the WRPF.  Since we usually assume $\DiscFac$ not far below 1 and
parameter values such that $\PatR \approx 1$, this condition is clearly not very
different from the \WRIC.

The upshot is that under these slightly stronger conditions the value
functions for the original problem define a contraction mapping with a
unique $\vFunc(\mRat)$.  But since $\lim_{n \rightarrow \infty}
\MinMinMPC_{T-n} = \MinMinMPC$ and $\lim_{n \rightarrow \infty}
\MaxMPC_{T-n} = \MaxMPC$, it must be the case that the $\vFunc(\mRat)$
toward which these $\vFunc_{T-n}$'s are converging is the {\it same}
$\vFunc(\mRat)$ that was the endpoint of the contraction defined by
our operator $\TMap$.  Thus, under our slightly stronger (but still
quite weak) conditions, not only do the value functions defined by
\eqref{\localorexternallabel{eq:veqn}} converge, they converge to the same unique $\vFunc$
defined by $\TMap$.\footnote{It seems likely that convergence of the
  value functions for the original problem could be proven even if
  only the \WRIC~were imposed; but that proof is not an essential part
  of the enterprise of this paper and is therefore left for future
  work.}

\onlyinsubfile{\bibliography{\LaTeXGenerated/BufferStockTheory,economics}}


\end{document}
% Local Variables:
% eval: (setq TeX-command-list  (remove '("Biber" "biber %s" TeX-run-Biber nil  (plain-tex-mode latex-mode doctex-mode ams-tex-mode texinfo-mode)  :help "Run Biber") TeX-command-list))
% eval: (setq TeX-command-list  (remove '("Biber" "biber %s" TeX-run-Biber nil  t  :help "Run Biber") TeX-command-list))
% tex-bibtex-command: "bibtex ../LaTeX/*"
% TeX-PDF-mode: t
% TeX-file-line-error: t
% TeX-debug-warnings: t
% LaTeX-command-style: (("" "%(PDF)%(latex) %(file-line-error) %(extraopts) -output-directory=../LaTeX %S%(PDFout)"))
% TeX-source-correlate-mode: t
% TeX-parse-self: t
% eval: (cond ((string-equal system-type "darwin") (progn (setq TeX-view-program-list '(("Skim" "/Applications/Skim.app/Contents/SharedSupport/displayline -b %n ../LaTeX/%o %b"))))))
% TeX-parse-all-errors: t
% End:
