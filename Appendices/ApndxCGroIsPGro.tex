% -*- enable-local-variables: :all; -*-
% LaTeX path to the root directory of the current project, from the directory in which this file resides
% and path to econtexPaths which defines the rest of the paths like \FigDir
\providecommand{\econtexRoot}{}\renewcommand{\econtexRoot}{.}
\providecommand{\econtexPaths}{}\renewcommand{\econtexPaths}{\econtexRoot/Resources/econtexPaths}
% The \commands below are required to allow sharing of the same base code via Github between TeXLive on a local machine and Overleaf (which is a proxy for "a standard distribution of LaTeX").  This is an ugly solution to the requirement that custom LaTeX packages be accessible, and that Overleaf seems to ignore symbolic links (even if they are relative links to valid locations)
\providecommand{\econtex}{\econtexRoot/Resources/texmf-local/tex/latex/econtex}
\providecommand{\econtexSetup}{\econtexRoot/Resources/texmf-local/tex/latex/econtexSetup}
\providecommand{\econtexShortcuts}{\econtexRoot/Resources/texmf-local/tex/latex/econtexShortcuts}
\providecommand{\econtexBibMake}{\econtexRoot/Resources/texmf-local/tex/latex/econtexBibMake}
\providecommand{\econtexBibStyle}{\econtexRoot/Resources/texmf-local/bibtex/bst/econtex}
\providecommand{\econtexBib}{economics}
\providecommand{\notes}{\econtexRoot/Resources/texmf-local/tex/latex/handout}
\providecommand{\handoutSetup}{\econtexRoot/Resources/texmf-local/tex/latex/handoutSetup}
\providecommand{\handoutShortcuts}{\econtexRoot/Resources/texmf-local/tex/latex/handoutShortcuts}
\providecommand{\handoutBibMake}{\econtexRoot/Resources/texmf-local/tex/latex/handoutBibMake}
\providecommand{\handoutBibStyle}{\econtexRoot/Resources/texmf-local/bibtex/bst/handout}

\providecommand{\FigDir}{\econtexRoot/Figures}
\providecommand{\CodeDir}{\econtexRoot/Code}
\providecommand{\DataDir}{\econtexRoot/Data}
\providecommand{\SlideDir}{\econtexRoot/Slides}
\providecommand{\TableDir}{\econtexRoot/Tables}
\providecommand{\ApndxDir}{\econtexRoot/Appendices}

\providecommand{\ResourcesDir}{\econtexRoot/Resources}
\ifnum\pdfshellescape=1
\providecommand{\rootFromOut}{..} % Path back to root directory from output-directory
\providecommand{\LaTeXGenerated}{\econtexRoot/LaTeX} % Put generated files in subdirectory
\providecommand{\EqDir}{\econtexRoot/Equations} % Put generated files in subdirectory
\else
\providecommand{\rootFromOut}{.} % Path back to root directory 
\providecommand{\LaTeXGenerated}{\econtexRoot/} % Put generated files in main directory (because not allowed in subdirectory)
\providecommand{\EqDir}{\econtexRoot/} % Put generated files in main directory
\fi
\providecommand{\econtexPaths}{\econtexRoot/Resources/econtexPaths}
\providecommand{\LaTeXInputs}{\econtexRoot/Resources/LaTeXInputs}

\documentclass[\econtexRoot/BufferStockTheory]{subfiles}
% LaTeX path to the root directory of the current project, from the directory in which this file resides
% and path to econtexPaths which defines the rest of the paths like \FigDir
\providecommand{\econtexRoot}{}\renewcommand{\econtexRoot}{.}
\providecommand{\econtexPaths}{}\renewcommand{\econtexPaths}{\econtexRoot/Resources/econtexPaths}
% The \commands below are required to allow sharing of the same base code via Github between TeXLive on a local machine and Overleaf (which is a proxy for "a standard distribution of LaTeX").  This is an ugly solution to the requirement that custom LaTeX packages be accessible, and that Overleaf seems to ignore symbolic links (even if they are relative links to valid locations)
\providecommand{\econtex}{\econtexRoot/Resources/texmf-local/tex/latex/econtex}
\providecommand{\econtexSetup}{\econtexRoot/Resources/texmf-local/tex/latex/econtexSetup}
\providecommand{\econtexShortcuts}{\econtexRoot/Resources/texmf-local/tex/latex/econtexShortcuts}
\providecommand{\econtexBibMake}{\econtexRoot/Resources/texmf-local/tex/latex/econtexBibMake}
\providecommand{\econtexBibStyle}{\econtexRoot/Resources/texmf-local/bibtex/bst/econtex}
\providecommand{\econtexBib}{economics}
\providecommand{\notes}{\econtexRoot/Resources/texmf-local/tex/latex/handout}
\providecommand{\handoutSetup}{\econtexRoot/Resources/texmf-local/tex/latex/handoutSetup}
\providecommand{\handoutShortcuts}{\econtexRoot/Resources/texmf-local/tex/latex/handoutShortcuts}
\providecommand{\handoutBibMake}{\econtexRoot/Resources/texmf-local/tex/latex/handoutBibMake}
\providecommand{\handoutBibStyle}{\econtexRoot/Resources/texmf-local/bibtex/bst/handout}

\providecommand{\FigDir}{\econtexRoot/Figures}
\providecommand{\CodeDir}{\econtexRoot/Code}
\providecommand{\DataDir}{\econtexRoot/Data}
\providecommand{\SlideDir}{\econtexRoot/Slides}
\providecommand{\TableDir}{\econtexRoot/Tables}
\providecommand{\ApndxDir}{\econtexRoot/Appendices}

\providecommand{\ResourcesDir}{\econtexRoot/Resources}
\ifnum\pdfshellescape=1
\providecommand{\rootFromOut}{..} % Path back to root directory from output-directory
\providecommand{\LaTeXGenerated}{\econtexRoot/LaTeX} % Put generated files in subdirectory
\providecommand{\EqDir}{\econtexRoot/Equations} % Put generated files in subdirectory
\else
\providecommand{\rootFromOut}{.} % Path back to root directory 
\providecommand{\LaTeXGenerated}{\econtexRoot/} % Put generated files in main directory (because not allowed in subdirectory)
\providecommand{\EqDir}{\econtexRoot/} % Put generated files in main directory
\fi
\providecommand{\econtexPaths}{\econtexRoot/Resources/econtexPaths}
\providecommand{\LaTeXInputs}{\econtexRoot/Resources/LaTeXInputs}

\onlyinsubfile{% https://tex.stackexchange.com/questions/463699/proper-reference-numbers-with-subfiles
    \csname @ifpackageloaded\endcsname{xr-hyper}{%
      \externaldocument{\econtexRoot/BufferStockTheory}% xr-hyper in use; optional argument for url of main.pdf for hyperlinks
    }{%
      \externaldocument{\econtexRoot/BufferStockTheory}% xr in use
    }%
    \renewcommand\labelprefix{}%
    % Initialize the counters via the labels belonging to the main document:
    \setcounter{equation}{\numexpr\getrefnumber{\labelprefix eq:Dummy}\relax}% eq:Dummy is the last number used for an equation in the main text; start counting up from there
}


\onlyinsubfile{\externaldocument{\LaTeXGenerated/BufferStockTheory}} % Get xrefs -- esp to appendix -- from main file; only works properly if main file has already been compiled;

\begin{document}

\section{Equality of Aggregate Consumption Growth and Income Growth with Transitory Shocks}\label{sec:ApndxCGroIsPGro}

Section \ref{subsec:cGroEqPGroQ} asserted that in the absence of permanent shocks it is possible to prove
that the growth factor for aggregate consumption approaches that for aggregate permanent
income.  This section establishes that result.

First define $\aFunc(\mRat)$ as the function that yields optimal end-of-period assets as a function of $\mRat$.

Suppose the population starts in period $t$ with an arbitrary value for
 $\mbox{cov}_{t}(a_{t+1,i},\pLevBF_{t+1,i})$. 
 Then if $\mSS$ is the invariant mean level of $\mRat$ we can define a `mean MPS away from $\mSS$' function
 \newcommand{\AMPS}{\bar{\acute{\aFunc}}}:
\begin{equation}\begin{gathered}\begin{aligned}
 \AMPS(\Delta)  & =  \Delta^{-1}\int_{\mSS}^{\mSS+\Delta} \aFunc^{\prime}(z)
 dz \label{eq:checkaFunc} \nonumber
\end{aligned}\end{gathered}\end{equation}
where the combination of the bar and the $\acute{}$ are meant to signify that this is the average value of the derivative over the interval.
Since $\pShk_{t+1,i}=1$, $\Rnorm_{t+1,i}$ is a constant at $\Rnorm$, if we define $\aSS$ as the value of $\aRat$ corresponding to $\mRat = \mSS$, we can write
\begin{equation}\begin{gathered}\begin{aligned}
  a_{t+1,i} 
& =   \aSS+(\mRat_{t+1,i}-\mSS)\AMPS(\overbrace{\Rnorm
    a_{t,i}+\tShkAll_{t+1,i}}^{\mRat_{t+1,i}}-\mSS) \nonumber
\end{aligned}\end{gathered}\end{equation}
so
\begin{equation}\begin{gathered}\begin{aligned}
\mbox{cov}_{t}(a_{t+1,i},\pLevBF_{t+1,i})
 & = \mbox{cov}_{t}\left(\AMPS(\Rnorm  a_{t,i}+\tShkAll_{t+1,i}-\mSS)
  ,\PGro   \pLevBF_{t,i}\right). \nonumber
\end{aligned}\end{gathered}\end{equation}

But since ${\Rfree}^{-1}(\pZero  \Rfree\DiscFac)^{1/\CRRA} < \AMPS(\mRat) < \PatR $,
\begin{equation}
  |\mbox{cov}_{t}((\pZero  \Rfree\DiscFac)^{1/\CRRA}a_{t+1,i},\pLevBF_{t+1,i})| <
  |\mbox{cov}_{t}(a_{t+1,i},\pLevBF_{t+1,i})| <
  |\mbox{cov}_{t}({\Pat}a_{t+1,i},\pLevBF_{t+1,i})| \nonumber
\end{equation}
and for the version of the model with no permanent shocks the \GIC~says that
${\Pat} < \PGro, $ while the {\FHWC} says that $\PGro < \Rfree$
\begin{equation}\begin{gathered}\begin{aligned}
  |\mbox{cov}_{t}(a_{t+1,i},\pLevBF_{t+1,i})| < \PGro
  |\mbox{cov}_{t}(a_{t,i},\pLevBF_{t,i})|. \nonumber
\end{aligned}\end{gathered}\end{equation}


This means that from any arbitrary starting value, the relative
size of the covariance term shrinks to zero over time (compared
to the $\ASS \PGro^{n}$ term which is growing steadily
by the factor $\PGro$).  Thus, $\lim_{n \rightarrow \infty} \ALevBF_{t+n+1}/\ALevBF_{t+n} = \PGro$.

This logic unfortunately does not go through when there are permanent
shocks, because the $\Rnorm _{t+1,i}$ terms are not independent
of the permanent income shocks.

To see the problem clearly, define $\breve{\Rnorm }=\Mean\left[\Rnorm _{t+1,i}\right]$ and consider a first order Taylor expansion of $\AMPS(\mRat_{t+1,i})$ around $\mTarg_{t+1,i}=\breve{\Rnorm } a_{t,i}+1,$
\begin{align*}
  \AMPS_{t+1,i} & \approx 
  \AMPS(\mTarg_{t+1,i})+\AMPS^{\prime}(\mTarg_{t+1,i})\left(\mRat_{t+1,i}-\mTarg_{t+1,i}\right)
 \nonumber.
\end{align*}


The problem comes from the $\AMPS^{\prime}$ term.  The
concavity of the consumption function implies convexity of the
$\aFunc$ function, so this term is strictly positive but we have no
theory to place bounds on its size as we do for its level $\AMPS$.
We cannot rule out by theory that a positive shock to permanent income (which has a
negative effect on $\mRat_{t+1,i}$) could have a (locally) unboundedly positive
effect on $\AMPS^{\prime}$ (as for instance if it pushes the
consumer arbitrarily close to the self-imposed liquidity constraint).


\end{document}

% Local Variables:
% eval: (setq TeX-command-list  (assq-delete-all (car (assoc "BibTeX" TeX-command-list)) TeX-command-list))
% eval: (setq TeX-command-list  (assq-delete-all (car (assoc "BibTeX" TeX-command-list)) TeX-command-list))
% eval: (setq TeX-command-list  (assq-delete-all (car (assoc "BibTeX" TeX-command-list)) TeX-command-list))
% eval: (setq TeX-command-list  (assq-delete-all (car (assoc "BibTeX" TeX-command-list)) TeX-command-list))
% eval: (setq TeX-command-list  (assq-delete-all (car (assoc "Biber"  TeX-command-list)) TeX-command-list))
% eval: (add-to-list 'TeX-command-list '("BibTeX" "bibtex ../LaTeX/%s" TeX-run-BibTeX nil t                                                                              :help "Run BibTeX") t)
% eval: (add-to-list 'TeX-command-list '("BibTeX" "bibtex ../LaTeX/%s" TeX-run-BibTeX nil (plain-tex-mode latex-mode doctex-mode ams-tex-mode texinfo-mode context-mode) :help "Run BibTeX") t)
% TeX-PDF-mode: t
% TeX-file-line-error: t
% TeX-debug-warnings: t
% LaTeX-command-style: (("" "%(PDF)%(latex) %(file-line-error) %(extraopts) -output-directory=../LaTeX %S%(PDFout)"))
% TeX-source-correlate-mode: t
% TeX-parse-self: t
% eval: (cond ((string-equal system-type "darwin") (progn (setq TeX-view-program-list '(("Skim" "/Applications/Skim.app/Contents/SharedSupport/displayline -b %n ../LaTeX/%o %b"))))))
% TeX-parse-all-errors: t
% End:
