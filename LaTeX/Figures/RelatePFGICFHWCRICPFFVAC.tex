\tikzset{node distance=5cm, auto, >=latex}
\begin{figure}[h]
  \hypertarget{RelatePFGICFHWCRICPFFVAC}{}
  \centerline{
    \begin{tikzpicture}
      \node (thorn) {$\Pat$};
      \node (gamma) [right of = thorn] {$\PGro$};
      \node (rfree) [below of = thorn]{$\mathsf{\Rfree}$};
      \node (pffvacFac) [right of = rfree]
      {$\Rfree^{1/\CRRA}\PGro^{1 - 1/\CRRA} $}; % \left(\equiv (\Rfree \PGro)^{1/\CRRA}\PGro\right)
      \draw[->] (thorn) to node {$\mathrm{\PFGIC}$} (gamma);
      \draw[->] (thorn) to node [swap] {$\mathrm{\RIC}$} (rfree);
      \draw[->] (thorn) to node [swap] {$\mathrm{\PFFVAC}$} (pffvacFac);
      \draw[->] (gamma) to node {$\mathrm{\FHWC}$} (pffvacFac);
      \draw[->] (pffvacFac) to node {$\mathrm{\FHWC}$} (rfree);
    \end{tikzpicture}
  }
  \caption{Relation of \PFGIC, \FHWC, \RIC, and \PFFVAC} \label{fig:RelatePFGICFHWCRICPFFVAC}
  \footnotesize{Arrows reflect the direction of the relationship; an arrowhead points to the larger of the two quantities being compared.  For example, the topmost arrow, pointing from $\Pat$ to $\PGro$ indicates that $\PGro > \Pat$.}
\end{figure}
