% LaTeX path to the root directory of the current project, from the directory in which this file resides
% and path to econtexPaths which defines the rest of the paths like \FigDir
\providecommand{\econtexRoot}{}\renewcommand{\econtexRoot}{.}
\providecommand{\econtexPaths}{}\renewcommand{\econtexPaths}{\econtexRoot/Resources/econtexPaths}
% The \commands below are required to allow sharing of the same base code via Github between TeXLive on a local machine and Overleaf (which is a proxy for "a standard distribution of LaTeX").  This is an ugly solution to the requirement that custom LaTeX packages be accessible, and that Overleaf seems to ignore symbolic links (even if they are relative links to valid locations)
\providecommand{\econtex}{\econtexRoot/Resources/texmf-local/tex/latex/econtex}
\providecommand{\econtexSetup}{\econtexRoot/Resources/texmf-local/tex/latex/econtexSetup}
\providecommand{\econtexShortcuts}{\econtexRoot/Resources/texmf-local/tex/latex/econtexShortcuts}
\providecommand{\econtexBibMake}{\econtexRoot/Resources/texmf-local/tex/latex/econtexBibMake}
\providecommand{\econtexBibStyle}{\econtexRoot/Resources/texmf-local/bibtex/bst/econtex}
\providecommand{\econtexBib}{economics}
\providecommand{\notes}{\econtexRoot/Resources/texmf-local/tex/latex/handout}
\providecommand{\handoutSetup}{\econtexRoot/Resources/texmf-local/tex/latex/handoutSetup}
\providecommand{\handoutShortcuts}{\econtexRoot/Resources/texmf-local/tex/latex/handoutShortcuts}
\providecommand{\handoutBibMake}{\econtexRoot/Resources/texmf-local/tex/latex/handoutBibMake}
\providecommand{\handoutBibStyle}{\econtexRoot/Resources/texmf-local/bibtex/bst/handout}

\providecommand{\FigDir}{\econtexRoot/Figures}
\providecommand{\CodeDir}{\econtexRoot/Code}
\providecommand{\DataDir}{\econtexRoot/Data}
\providecommand{\SlideDir}{\econtexRoot/Slides}
\providecommand{\TableDir}{\econtexRoot/Tables}
\providecommand{\ApndxDir}{\econtexRoot/Appendices}

\providecommand{\ResourcesDir}{\econtexRoot/Resources}
\ifnum\pdfshellescape=1
\providecommand{\rootFromOut}{..} % Path back to root directory from output-directory
\providecommand{\LaTeXGenerated}{\econtexRoot/LaTeX} % Put generated files in subdirectory
\providecommand{\EqDir}{\econtexRoot/Equations} % Put generated files in subdirectory
\else
\providecommand{\rootFromOut}{.} % Path back to root directory 
\providecommand{\LaTeXGenerated}{\econtexRoot/} % Put generated files in main directory (because not allowed in subdirectory)
\providecommand{\EqDir}{\econtexRoot/} % Put generated files in main directory
\fi
\providecommand{\econtexPaths}{\econtexRoot/Resources/econtexPaths}
\providecommand{\LaTeXInputs}{\econtexRoot/Resources/LaTeXInputs}

\documentclass[\econtexRoot/BufferStockTheory]{subfiles}
% LaTeX path to the root directory of the current project, from the directory in which this file resides
% and path to econtexPaths which defines the rest of the paths like \FigDir
\providecommand{\econtexRoot}{}\renewcommand{\econtexRoot}{.}
\providecommand{\econtexPaths}{}\renewcommand{\econtexPaths}{\econtexRoot/Resources/econtexPaths}
% The \commands below are required to allow sharing of the same base code via Github between TeXLive on a local machine and Overleaf (which is a proxy for "a standard distribution of LaTeX").  This is an ugly solution to the requirement that custom LaTeX packages be accessible, and that Overleaf seems to ignore symbolic links (even if they are relative links to valid locations)
\providecommand{\econtex}{\econtexRoot/Resources/texmf-local/tex/latex/econtex}
\providecommand{\econtexSetup}{\econtexRoot/Resources/texmf-local/tex/latex/econtexSetup}
\providecommand{\econtexShortcuts}{\econtexRoot/Resources/texmf-local/tex/latex/econtexShortcuts}
\providecommand{\econtexBibMake}{\econtexRoot/Resources/texmf-local/tex/latex/econtexBibMake}
\providecommand{\econtexBibStyle}{\econtexRoot/Resources/texmf-local/bibtex/bst/econtex}
\providecommand{\econtexBib}{economics}
\providecommand{\notes}{\econtexRoot/Resources/texmf-local/tex/latex/handout}
\providecommand{\handoutSetup}{\econtexRoot/Resources/texmf-local/tex/latex/handoutSetup}
\providecommand{\handoutShortcuts}{\econtexRoot/Resources/texmf-local/tex/latex/handoutShortcuts}
\providecommand{\handoutBibMake}{\econtexRoot/Resources/texmf-local/tex/latex/handoutBibMake}
\providecommand{\handoutBibStyle}{\econtexRoot/Resources/texmf-local/bibtex/bst/handout}

\providecommand{\FigDir}{\econtexRoot/Figures}
\providecommand{\CodeDir}{\econtexRoot/Code}
\providecommand{\DataDir}{\econtexRoot/Data}
\providecommand{\SlideDir}{\econtexRoot/Slides}
\providecommand{\TableDir}{\econtexRoot/Tables}
\providecommand{\ApndxDir}{\econtexRoot/Appendices}

\providecommand{\ResourcesDir}{\econtexRoot/Resources}
\ifnum\pdfshellescape=1
\providecommand{\rootFromOut}{..} % Path back to root directory from output-directory
\providecommand{\LaTeXGenerated}{\econtexRoot/LaTeX} % Put generated files in subdirectory
\providecommand{\EqDir}{\econtexRoot/Equations} % Put generated files in subdirectory
\else
\providecommand{\rootFromOut}{.} % Path back to root directory 
\providecommand{\LaTeXGenerated}{\econtexRoot/} % Put generated files in main directory (because not allowed in subdirectory)
\providecommand{\EqDir}{\econtexRoot/} % Put generated files in main directory
\fi
\providecommand{\econtexPaths}{\econtexRoot/Resources/econtexPaths}
\providecommand{\LaTeXInputs}{\econtexRoot/Resources/LaTeXInputs}

\onlyinsubfile{% https://tex.stackexchange.com/questions/463699/proper-reference-numbers-with-subfiles
    \csname @ifpackageloaded\endcsname{xr-hyper}{%
      \externaldocument{\econtexRoot/BufferStockTheory}% xr-hyper in use; optional argument for url of main.pdf for hyperlinks
    }{%
      \externaldocument{\econtexRoot/BufferStockTheory}% xr in use
    }%
    \renewcommand\labelprefix{}%
    % Initialize the counters via the labels belonging to the main document:
    \setcounter{equation}{\numexpr\getrefnumber{\labelprefix eq:Dummy}\relax}% eq:Dummy is the last number used for an equation in the main text; start counting up from there
}



\onlyinsubfile{\externaldocument{\LaTeXGenerated/BufferStockTheory}} % Get xrefs -- esp to appendix -- from main file; only works properly if main file has already been compiled;
\begin{document}

  \section{The Perfect Foresight Liquidity Constrained Solution as a Limit}
\label{sec:LiqConstrAsLimit}

Formally, suppose we change the description of the problem by making
the following two assumptions:
\begin{eqnarray*}
    \pZero   & = 0
\\  c_{t} & \leq  \mRat_{t} \label{eq:liqconstr},
\end{eqnarray*}
and we designate the solution to this consumer's problem
$\constr{\cFunc}_{t}(\mRat)$.  We will henceforth refer to this as the
problem of the `restrained' consumer (and, to avoid a common
confusion, we will refer to the consumer as `constrained' only in
circumstances when the constraint is actually binding).

Redesignate the consumption function that emerges from our original
problem for a given fixed $\pZero$ as $\cFunc_{t}(\mRat;\pZero)$ where we
separate the arguments by a semicolon to distinguish between $\mRat$,
which is a state variable, and $\pZero$, which is not.  The
proposition we wish to demonstrate is
\begin{equation}\begin{gathered}\begin{aligned}
  \lim_{\pZero \downarrow 0} \cFunc_{t}(\mRat;\pZero)  & = \constr{\cFunc}_{t}(\mRat). \label{eq:RestrEqUnrestr} 
\end{aligned}\end{gathered}\end{equation}

We will first examine the problem in period $T-1$, then
argue that the desired result propagates to earlier periods.
For simplicity, suppose that the interest, growth, and time-preference
factors are $\DiscFac = \Rfree = \PGro = 1,$ and there are no permanent
shocks, $\pShk=1$; the results below are easily generalized
to the full-fledged version of the problem.

The solution to the restrained consumer's optimization problem can be
obtained as follows.  Assuming that the consumer's behavior in period
$T$ is given by $\cFunc_{T}(\mRat)$ (in practice, this will be
$\cFunc_{T}(\mRat)=m$), consider the unrestrained optimization problem
\begin{equation}\begin{gathered}\begin{aligned}
  \constr{\aFunc}^{*}_{T-1}(\mRat)  & = \underset{\aRat}{\arg \max} \left\{\uFunc(\mRat-\aRat) +  \int_{\underline{\tShkEmp}}^{\bar{\tShkEmp}} \vFunc_{T}(a+\tShkEmp) d\CDF_{\tShkEmp} \right\}. \label{eq:vUnconstr}
\end{aligned}\end{gathered}\end{equation}

As usual, the envelope theorem tells us that
$\vFunc_{T}^{\prime}(\mRat)=\uP(\cFunc_{T}(\mRat))$ so the expected marginal
value of ending period $T-1$ with assets $\aRat$ can be defined as
\begin{equation}\begin{gathered}\begin{aligned}
  \constr{\mathfrak{v}}_{T-1}^{\prime}(a)  & \equiv  \int_{\underline{\tShkEmp}}^{\bar{\tShkEmp}} \uP(\cFunc_{T}(a+\tShkEmp)) d\CDF_{\tShkEmp}, \notag
\end{aligned}\end{gathered}\end{equation}
and the solution to \eqref{\localorexternallabel{eq:vUnconstr}} will satisfy
\begin{equation}\begin{gathered}\begin{aligned}
  \uP(\mRat-\aRat)  & =  \constr{\mathfrak{v}}_{T-1}^{\prime}(a) \label{eq:uPConstr}.
%
\end{aligned}\end{gathered}\end{equation}

$\constr{\aFunc}_{T-1}^{*}(\mRat)$ therefore answers the question ``With what
level of assets would the restrained consumer like to end period $T-1$
if the constraint $c_{T-1} \leq \mRat_{T-1}$ did not exist?''  (Note that
the restrained consumer's income process remains different from the
process for the unrestrained consumer so long as $\pZero>0$.)  The
restrained consumer's actual asset position will be
 \begin{equation}\begin{gathered}\begin{aligned}
  \constr{\aFunc}_{T-1}(\mRat)  & = \max[0,\constr{\aFunc}^{*}_{T-1}(\mRat)], \notag
\end{aligned}\end{gathered}\end{equation}
reflecting the inability of the restrained consumer to spend more than
current resources, and note (as pointed out by
Deaton~\citeyearpar{deatonLiqConstr}) that
 \begin{equation}\begin{gathered}\begin{aligned}
  \mRat^{1}_{\#}  & = \left( \constr{\mathfrak{v}}_{T-1}^{\prime}(0)\right)^{-1/\CRRA} \notag
 \end{aligned}\end{gathered}\end{equation}
is the cusp value of $\mRat$ at which the constraint makes the
transition between binding and non-binding in period $T-1$.

Analogously to \eqref{\localorexternallabel{eq:uPConstr}}, defining
\begin{equation}\begin{gathered}\begin{aligned}
  \mathfrak{v}_{T-1}^{\prime}(a;\pZero)  & \equiv  \left[\pZero \aRat^{-\CRRA}+\pNotZero\int_{\underline{\tShkEmp}}^{\bar{\tShkEmp}} \left(\cFunc_{T}(a+\tShkEmp/\pNotZero)\right)^{-\CRRA} d\CDF_{\tShkEmp}\right], \label{eq:vFrakPrime}
\end{aligned}\end{gathered}\end{equation}
the Euler equation for the original consumer's problem implies
\begin{equation}\begin{gathered}\begin{aligned}
 (\mRat-\aRat)^{-\CRRA}  & = \mathfrak{v}_{T-1}^{\prime}(a;\pZero) \label{eq:uPUnconstr}
\end{aligned}\end{gathered}\end{equation}
with solution $\aFunc_{T-1}^{*}(\mRat;\pZero)$.  Now note that for any
fixed $\aRat>0$, $\lim_{\pZero \downarrow 0}
\mathfrak{v}_{T-1}^{\prime}(a;\pZero) =
\constr{\mathfrak{v}}_{T-1}^{\prime}(a)$.  Since the LHS of
\eqref{\localorexternallabel{eq:uPConstr}} and \eqref{\localorexternallabel{eq:uPUnconstr}} are identical, this
means that $\lim_{\pZero \downarrow 0}
\aFunc_{T-1}^{*}(\mRat;\pZero) =
\constr{\aFunc}_{T-1}^{*}(\mRat)$.  That is, for any fixed value of
$\mRat > \mRat^{1}_{\#}$ such that the consumer subject to the
restraint would voluntarily choose to end the period with positive
assets, the level of end-of-period assets for the unrestrained
consumer approaches the level for the restrained consumer as $\pZero
\downarrow 0$.  With the same $\aRat$ and the same $\mRat$, the
consumers must have the same $c$, so the consumption functions are
identical in the limit.

Now consider values $\mRat\leq \mRat^{1}_{\#}$ for which the restrained consumer
is constrained.  It is obvious that the baseline consumer will never
choose $\aRat \leq 0$ because the first term in \eqref{\localorexternallabel{eq:vFrakPrime}} is $\lim_{\aRat \downarrow 0} \pZero  \aRat^{-\CRRA} =
\infty$, while $\lim_{\aRat \downarrow 0} (\mRat-\aRat)^{-\CRRA}$ is finite (the
marginal value of end-of-period assets approaches infinity as assets approach
zero, but the marginal utility of consumption has a finite limit for $\mRat>0$).
The subtler question is whether it is possible to rule out strictly positive
$\aRat$ for the unrestrained consumer.

The answer is yes.  Suppose, for some $\mRat < \mRat^{1}_{\#},$ that the
unrestrained consumer is considering ending the period with any
positive amount of assets $\aRat=\delta > 0$.  For any such $\delta$ we
have that $\lim_{\pZero  \downarrow 0} \mathfrak{v}_{T-1}^{\prime}(a;\pZero)=\constr{\mathfrak{v}}_{T-1}^{\prime}(a)$.
But by assumption we are considering a set of circumstances in which
$\constr{\aFunc}_{T-1}^{*}(\mRat) < 0$, and we showed earlier that
$\lim_{\pZero  \downarrow 0} \aFunc_{T-1}^{*}(\mRat;\pZero) = \constr{\aFunc}_{T-1}^{*}(\mRat)$.  So,
having assumed $\aRat = \delta > 0$, we have proven that the consumer
would optimally choose $\aRat < 0$, which is a contradiction.  A similar
argument holds for $\mRat = \mRat^{1}_{\#}$.

These arguments demonstrate that for any $\mRat>0$, $\lim_{\pZero
  \downarrow 0} \cFunc_{T-1}(\mRat;\pZero) =
\constr{\cFunc}_{T-1}(\mRat)$ which is the period $T-1$ version of
\eqref{\localorexternallabel{eq:RestrEqUnrestr}}.  But given equality of the period $T-1$
consumption functions, backwards recursion of the same arguments
demonstrates that the limiting consumption functions in previous
periods are also identical to the constrained function.

Note finally that another intuitive confirmation of the equivalence
between the two problems is that our formula~\eqref{\localorexternallabel{eq:MaxMPCDef}} for the maximal marginal
propensity to consume satisfies
\begin{eqnarray*}
  \lim_{\pZero \downarrow 0} \MaxMPC  & = 1,
\end{eqnarray*}
which makes sense because the marginal propensity to consume for a
constrained restrained consumer is 1 by our definitions of
`constrained' and `restrained.'

\onlyinsubfile{\bibliography{\LaTeXGenerated/BufferStockTheory,\econtexBib/economics}}

\end{document}
