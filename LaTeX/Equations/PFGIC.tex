\begin{align}
  \label{eq:PFGIC}
  \PatPGro &  < 1
\end{align}
\begin{align}
  \label{eq:PFGIC}
  \PatPGro &  < 1
\end{align}
\begin{align}
  \label{eq:PFGIC}
  \PatPGro &  < 1
\end{align}
\begin{align}
  \label{eq:PFGIC}
  \PatPGro &  < 1
\end{align}
\input{\EqDir/PFGIC}
which is equivalent to \eqref{eq:LiqConstrBinds} (exponentiate both
sides by $1/\CRRA$).

If the \RIC~\eqref{eq:RIC} and the \FHWC~\eqref{eq:FHWC} hold, appendix \ref{sec:ApndxLiqConstr} shows
that, for some $0 < \mRat_{\#} < 1$, an unconstrained consumer behaving according to
\eqref{eq:cFuncPFUnc} would choose $\cRat < \mRat$ for all $\mRat >
\mRat_{\#}$.  In this case the solution to the constrained consumer's problem is simple: For any $\mRat
\geq \mRat_{\#}$ the constraint does not bind (and will never bind in
the future) so the constrained consumption function is identical
to the unconstrained one.  If the consumer were somehow\footnote{``Somehow'' because $\mRat<1$ could only be
  obtained by entering the period with $\bRat < 0$ which the constraint
  forbids.}
to arrive at an $\mRat < \mRat_{\#} < 1$ the constraint would bind and
the consumer would have to consume $\cRat=\mRat$.  The $\circ$ accent will designate the limiting
constrained consumption function:
\begin{equation}
  \mathring{\cFunc}(\mRat)=
  \begin{cases}
    \mRat & \text{if $\mRat < \mRat_{\#}$} \\
    \bar{\cFunc}(\mRat)  & \text{if $\mRat \geq \mRat_{\#}$.}
  \end{cases}
\end{equation}

More useful is the case where the perfect foresight growth and return impatience conditions both hold.  In this case appendix \ref{sec:ApndxLiqConstr} shows that the limiting constrained consumption function is piecewise linear, with $\mathring{\cFunc}(\mRat)=\mRat$ up to a first `kink point' at $\mRat_{\#}^{1}>1$, and with discrete declines in the MPC at a set of kink points $\{\mRat_{\#}^{1},\mRat_{\#}^{2},...\}$.  As $\mRat \uparrow \infty$ the constrained consumption function $\mathring{\cFunc}(\mRat)$ becomes arbitrarily close to the unconstrained $\bar{\cFunc}(\mRat)$, and the marginal propensity to consume function $\mathring{\MPCFunc}(\mRat) \equiv \mathring{\cFunc}^{\prime}(\mRat)$ limits to $\MinMPC$.  Similarly, the value function $\mathring{\vFunc}(\mRat)$ is nondegenerate and limits into the value function of the unconstrained consumer.  This logic holds even when the finite human wealth condition fails (denoted \cncl{\FHWC}):  A solution exists because the constraint prevents the consumer from borrowing against infinite human wealth to finance infinite current consumption.  Under these circumstances, the consumer who starts with any amount of resources $\bRat_{t} > 1$ will, over time, run those resources down so that by some finite number of periods $n$ in the future the consumer will reach $\bRat_{t+n} = 0$, and thereafter will set $\cRat = \mRat = 1$ for eternity, a policy that will (using \eqref{eq:ValuePFAnalytical}) yield value of \hypertarget{PFFVAC}{}
\begin{align}
  \vLevBF_{t+n}  & = \uFunc(\pLevBF_{t+n})\left(1+\DiscFac
                   \PGro^{1-\CRRA}+(\DiscFac \PGro^{1-\CRRA})^{2}+...\right) \notag
  \\  & = \PGro^{n(1-\CRRA)} \uFunc(\pLevBF_{t})\left(\frac{1-(\DiscFac
        \PGro^{1-\CRRA})^{T-(t+n)+1}}{1-\DiscFac \PGro^{1-\CRRA}}\right),
        \notag
\end{align}
which will be finite whenever
\begin{align}
  \overbrace{\DiscFac \PGro^{1-\CRRA} }^{\equiv \DiscAlt}  & < 1 \nonumber
  \\ \DiscFac \Rfree \PGro^{-\CRRA}  & < \Rfree/\PGro \label{eq:PFFVAC}
  \\ \PatPGro  & < (\Rfree/\PGro)^{1/\CRRA}, \nonumber
\end{align}
where we call $\DiscAlt$ the `Perfect Foresight Value Of Autarky Factor' ({\PFFVAF})and which we call the Perfect Foresight Finite Value of Autarky Condition, \PFFVAC, because it guarantees that a consumer who always spends all permanent income will have finite value (the consumer has `finite autarky value').  Note that the version of the \PFFVAC~in \eqref{eq:PFFVAC} implies the \PFGIC~$\PatPGro < 1$ whenever \cncl{\FHWC} ($\Rfree < \PGro$) holds.  So, if \cncl{\FHWC}, value for any finite $\mRat$ will be the sum of two finite numbers: The component due to the unconstrained consumption choice made over the finite horizon leading up to $\bRat_{t+n} = 0$, and the finite component due to the value of consuming all $\pLevBF_{t+n}$ thereafter.  The consumer's value function is therefore nondegenerate.

\hypertarget{RICandFHWCFail}{}
The most peculiar possibility occurs when the \RIC~fails.  Under these circumstances the \FHWC~must also fail (Appendix~\ref{sec:ApndxLiqConstr}), and the constrained consumption function is nondegenerate.  (See appendix Figure~\ref{fig:PFGICHoldsFHWCFailsRICFails} for a numerical example).  While it is true that $\lim_{m \uparrow \infty}
\mathring{\MPCFunc}(\mRat) = 0$, nevertheless the limiting constrained
consumption function $\mathring{\cFunc}(\mRat)$ is strictly positive
and strictly increasing in $\mRat$.  This result interestingly
reconciles the conflicting intuitions from the unconstrained case,
where \cncl{\RIC} would suggest a degenerate limit of
$\mathring{\cFunc}(\mRat)=0$ while \cncl{\FHWC} would suggest a
degenerate limit of $\mathring{\cFunc}(\mRat)=\infty$.

Table~\ref{table:Comparison} and \ref{table:Required} (and appendix table~\ref{table:LiqConstrScenarios}) codify the key points to help the reader keep them straight.


An intuitive representation of the relations of the conditions is presented in Figure~\ref{fig:RelatePFGICFHWCRICPFFVAC} (for the perfect foresight case).  Each node represents a quantity considered in the foregoing analysis.  The arrow associated with each inequality condition reflects the imposition of that condition.  For example, one way of writing the {\PFFVAC} is $\Pat < \Rfree^{1/\CRRA} \PGro^{1-1/\CRRA}$, and this is captured by the diagonal arrow.  Traversing the diagram clockwise starting at $\Pat$ involves imposing first the {\PFGIC} then the {\FHWC}, and the consequent arrival at the bottom right node tells us that these two conditions jointly imply that the {\PFFVAC} holds.  Reversal of a condition will reverse the arrow's direction; so, for example, {\cncl{\FHWC}} would correspond to reversing the directon of the bottommost arrow.  This would allow us to follow the arrows in a counterclockwise direction from $\Pat$ to the bottom right node, leading to the conclusion that the combination of {\RIC} and \cncl{\FHWC} implies that the {\PFFVAC} holds.  (Readers unfamiliar with diagrams of this kind may wish to consult the expanded discussion in Appendix~\ref{sec:ApndxConditionDiagrams}).

\providecommand{\figName}{RelatePFGICFHWCRICPFFVAC} % Allows generic definition of hypertargets based on title of figure
\providecommand{\figFile}{\figName} %  and on filename
%\hypertarget{\figFile}{}
\hypertarget{\figName}{}
\input{\FigDir/\figName} % Read in the tex to generate the figure

We now turn to the case with uncertainty.  The model without constraints but with uncertainty will turn out to be a close parallel to the model with constraints but without uncertainty.

\hypertarget{Uncertainty-Modified-Conditions}{}
\subsection{Uncertainty-Modified Conditions}
\subsubsection{Impatience}

When uncertainty is introduced, the expectation of $\bRat_{t+1}$ can be rewritten as:
\begin{align}
  \Ex_{t}[\bRat_{t+1}]  & =  \aRat_{t}\Ex_{t}[\Rnorm_{t+1}] = \aRat_{t}\Rnorm\Ex_{t}[\pShk_{t+1}^{-1}] \label{eq:EbRat}
\end{align}
where Jensen's inequality guarantees that the expectation of the inverse of the permanent shock is strictly greater than one.  It will be convenient to define the object \hypertarget{InvEpShkInv}{}
\begin{align*}
  \InvEpShkInv  & \equiv  (\EpShkInv)^{-1}
\end{align*}
because this permits us to write expressions like the RHS of
\eqref{eq:EbRat} compactly as, e.g., $\aRat_{t}\Rnorm/
\InvEpShkInv^{-1}.$\footnote{One way to think of $\InvEpShkInv$ is as
  a particular kind of a `certainty equivalent' of the shock; this
  captures the intuition that mean-one shock renders a given mean
  level of income less valuable than if the shock did not exist, so
  that $\InvEpShkInv < 1$.}  We refer to this as the `compensated return,' because it compensates (in a risk-neutral way) for the effect of
uncertainty on the expected growth-normalized return (in the sense implicitly defined in
\eqref{eq:EbRat}).

\hypertarget{GIC}{}
\hypertarget{GICI}{}
We can now transparently generalize the \PFGIC~\eqref{eq:PFGIC} by defining a `compensated growth factor' \hypertarget{PGroAdj}{}
\begin{verbatimwrite}{\EqDir/PGroAdj}
  \begin{align}
    \PGroAdj  & =  \PGro \InvEpShkInv \label{eq:PGroAdj}
  \end{align}

which is equivalent to \eqref{eq:LiqConstrBinds} (exponentiate both
sides by $1/\CRRA$).

If the \RIC~\eqref{eq:RIC} and the \FHWC~\eqref{eq:FHWC} hold, appendix \ref{sec:ApndxLiqConstr} shows
that, for some $0 < \mRat_{\#} < 1$, an unconstrained consumer behaving according to
\eqref{eq:cFuncPFUnc} would choose $\cRat < \mRat$ for all $\mRat >
\mRat_{\#}$.  In this case the solution to the constrained consumer's problem is simple: For any $\mRat
\geq \mRat_{\#}$ the constraint does not bind (and will never bind in
the future) so the constrained consumption function is identical
to the unconstrained one.  If the consumer were somehow\footnote{``Somehow'' because $\mRat<1$ could only be
  obtained by entering the period with $\bRat < 0$ which the constraint
  forbids.}
to arrive at an $\mRat < \mRat_{\#} < 1$ the constraint would bind and
the consumer would have to consume $\cRat=\mRat$.  The $\circ$ accent will designate the limiting
constrained consumption function:
\begin{equation}
  \mathring{\cFunc}(\mRat)=
  \begin{cases}
    \mRat & \text{if $\mRat < \mRat_{\#}$} \\
    \bar{\cFunc}(\mRat)  & \text{if $\mRat \geq \mRat_{\#}$.}
  \end{cases}
\end{equation}

More useful is the case where the perfect foresight growth and return impatience conditions both hold.  In this case appendix \ref{sec:ApndxLiqConstr} shows that the limiting constrained consumption function is piecewise linear, with $\mathring{\cFunc}(\mRat)=\mRat$ up to a first `kink point' at $\mRat_{\#}^{1}>1$, and with discrete declines in the MPC at a set of kink points $\{\mRat_{\#}^{1},\mRat_{\#}^{2},...\}$.  As $\mRat \uparrow \infty$ the constrained consumption function $\mathring{\cFunc}(\mRat)$ becomes arbitrarily close to the unconstrained $\bar{\cFunc}(\mRat)$, and the marginal propensity to consume function $\mathring{\MPCFunc}(\mRat) \equiv \mathring{\cFunc}^{\prime}(\mRat)$ limits to $\MinMPC$.  Similarly, the value function $\mathring{\vFunc}(\mRat)$ is nondegenerate and limits into the value function of the unconstrained consumer.  This logic holds even when the finite human wealth condition fails (denoted \cncl{\FHWC}):  A solution exists because the constraint prevents the consumer from borrowing against infinite human wealth to finance infinite current consumption.  Under these circumstances, the consumer who starts with any amount of resources $\bRat_{t} > 1$ will, over time, run those resources down so that by some finite number of periods $n$ in the future the consumer will reach $\bRat_{t+n} = 0$, and thereafter will set $\cRat = \mRat = 1$ for eternity, a policy that will (using \eqref{eq:ValuePFAnalytical}) yield value of \hypertarget{PFFVAC}{}
\begin{align}
  \vLevBF_{t+n}  & = \uFunc(\pLevBF_{t+n})\left(1+\DiscFac
                   \PGro^{1-\CRRA}+(\DiscFac \PGro^{1-\CRRA})^{2}+...\right) \notag
  \\  & = \PGro^{n(1-\CRRA)} \uFunc(\pLevBF_{t})\left(\frac{1-(\DiscFac
        \PGro^{1-\CRRA})^{T-(t+n)+1}}{1-\DiscFac \PGro^{1-\CRRA}}\right),
        \notag
\end{align}
which will be finite whenever
\begin{align}
  \overbrace{\DiscFac \PGro^{1-\CRRA} }^{\equiv \DiscAlt}  & < 1 \nonumber
  \\ \DiscFac \Rfree \PGro^{-\CRRA}  & < \Rfree/\PGro \label{eq:PFFVAC}
  \\ \PatPGro  & < (\Rfree/\PGro)^{1/\CRRA}, \nonumber
\end{align}
where we call $\DiscAlt$ the `Perfect Foresight Value Of Autarky Factor' ({\PFFVAF})and which we call the Perfect Foresight Finite Value of Autarky Condition, \PFFVAC, because it guarantees that a consumer who always spends all permanent income will have finite value (the consumer has `finite autarky value').  Note that the version of the \PFFVAC~in \eqref{eq:PFFVAC} implies the \PFGIC~$\PatPGro < 1$ whenever \cncl{\FHWC} ($\Rfree < \PGro$) holds.  So, if \cncl{\FHWC}, value for any finite $\mRat$ will be the sum of two finite numbers: The component due to the unconstrained consumption choice made over the finite horizon leading up to $\bRat_{t+n} = 0$, and the finite component due to the value of consuming all $\pLevBF_{t+n}$ thereafter.  The consumer's value function is therefore nondegenerate.

\hypertarget{RICandFHWCFail}{}
The most peculiar possibility occurs when the \RIC~fails.  Under these circumstances the \FHWC~must also fail (Appendix~\ref{sec:ApndxLiqConstr}), and the constrained consumption function is nondegenerate.  (See appendix Figure~\ref{fig:PFGICHoldsFHWCFailsRICFails} for a numerical example).  While it is true that $\lim_{m \uparrow \infty}
\mathring{\MPCFunc}(\mRat) = 0$, nevertheless the limiting constrained
consumption function $\mathring{\cFunc}(\mRat)$ is strictly positive
and strictly increasing in $\mRat$.  This result interestingly
reconciles the conflicting intuitions from the unconstrained case,
where \cncl{\RIC} would suggest a degenerate limit of
$\mathring{\cFunc}(\mRat)=0$ while \cncl{\FHWC} would suggest a
degenerate limit of $\mathring{\cFunc}(\mRat)=\infty$.

Table~\ref{table:Comparison} and \ref{table:Required} (and appendix table~\ref{table:LiqConstrScenarios}) codify the key points to help the reader keep them straight.


An intuitive representation of the relations of the conditions is presented in Figure~\ref{fig:RelatePFGICFHWCRICPFFVAC} (for the perfect foresight case).  Each node represents a quantity considered in the foregoing analysis.  The arrow associated with each inequality condition reflects the imposition of that condition.  For example, one way of writing the {\PFFVAC} is $\Pat < \Rfree^{1/\CRRA} \PGro^{1-1/\CRRA}$, and this is captured by the diagonal arrow.  Traversing the diagram clockwise starting at $\Pat$ involves imposing first the {\PFGIC} then the {\FHWC}, and the consequent arrival at the bottom right node tells us that these two conditions jointly imply that the {\PFFVAC} holds.  Reversal of a condition will reverse the arrow's direction; so, for example, {\cncl{\FHWC}} would correspond to reversing the directon of the bottommost arrow.  This would allow us to follow the arrows in a counterclockwise direction from $\Pat$ to the bottom right node, leading to the conclusion that the combination of {\RIC} and \cncl{\FHWC} implies that the {\PFFVAC} holds.  (Readers unfamiliar with diagrams of this kind may wish to consult the expanded discussion in Appendix~\ref{sec:ApndxConditionDiagrams}).

\providecommand{\figName}{RelatePFGICFHWCRICPFFVAC} % Allows generic definition of hypertargets based on title of figure
\providecommand{\figFile}{\figName} %  and on filename
%\hypertarget{\figFile}{}
\hypertarget{\figName}{}
\input{\FigDir/\figName} % Read in the tex to generate the figure

We now turn to the case with uncertainty.  The model without constraints but with uncertainty will turn out to be a close parallel to the model with constraints but without uncertainty.

\hypertarget{Uncertainty-Modified-Conditions}{}
\subsection{Uncertainty-Modified Conditions}
\subsubsection{Impatience}

When uncertainty is introduced, the expectation of $\bRat_{t+1}$ can be rewritten as:
\begin{align}
  \Ex_{t}[\bRat_{t+1}]  & =  \aRat_{t}\Ex_{t}[\Rnorm_{t+1}] = \aRat_{t}\Rnorm\Ex_{t}[\pShk_{t+1}^{-1}] \label{eq:EbRat}
\end{align}
where Jensen's inequality guarantees that the expectation of the inverse of the permanent shock is strictly greater than one.  It will be convenient to define the object \hypertarget{InvEpShkInv}{}
\begin{align*}
  \InvEpShkInv  & \equiv  (\EpShkInv)^{-1}
\end{align*}
because this permits us to write expressions like the RHS of
\eqref{eq:EbRat} compactly as, e.g., $\aRat_{t}\Rnorm/
\InvEpShkInv^{-1}.$\footnote{One way to think of $\InvEpShkInv$ is as
  a particular kind of a `certainty equivalent' of the shock; this
  captures the intuition that mean-one shock renders a given mean
  level of income less valuable than if the shock did not exist, so
  that $\InvEpShkInv < 1$.}  We refer to this as the `compensated return,' because it compensates (in a risk-neutral way) for the effect of
uncertainty on the expected growth-normalized return (in the sense implicitly defined in
\eqref{eq:EbRat}).

\hypertarget{GIC}{}
\hypertarget{GICI}{}
We can now transparently generalize the \PFGIC~\eqref{eq:PFGIC} by defining a `compensated growth factor' \hypertarget{PGroAdj}{}
\begin{verbatimwrite}{\EqDir/PGroAdj}
  \begin{align}
    \PGroAdj  & =  \PGro \InvEpShkInv \label{eq:PGroAdj}
  \end{align}

which is equivalent to \eqref{eq:LiqConstrBinds} (exponentiate both
sides by $1/\CRRA$).

If the \RIC~\eqref{eq:RIC} and the \FHWC~\eqref{eq:FHWC} hold, appendix \ref{sec:ApndxLiqConstr} shows
that, for some $0 < \mRat_{\#} < 1$, an unconstrained consumer behaving according to
\eqref{eq:cFuncPFUnc} would choose $\cRat < \mRat$ for all $\mRat >
\mRat_{\#}$.  In this case the solution to the constrained consumer's problem is simple: For any $\mRat
\geq \mRat_{\#}$ the constraint does not bind (and will never bind in
the future) so the constrained consumption function is identical
to the unconstrained one.  If the consumer were somehow\footnote{``Somehow'' because $\mRat<1$ could only be
  obtained by entering the period with $\bRat < 0$ which the constraint
  forbids.}
to arrive at an $\mRat < \mRat_{\#} < 1$ the constraint would bind and
the consumer would have to consume $\cRat=\mRat$.  The $\circ$ accent will designate the limiting
constrained consumption function:
\begin{equation}
  \mathring{\cFunc}(\mRat)=
  \begin{cases}
    \mRat & \text{if $\mRat < \mRat_{\#}$} \\
    \bar{\cFunc}(\mRat)  & \text{if $\mRat \geq \mRat_{\#}$.}
  \end{cases}
\end{equation}

More useful is the case where the perfect foresight growth and return impatience conditions both hold.  In this case appendix \ref{sec:ApndxLiqConstr} shows that the limiting constrained consumption function is piecewise linear, with $\mathring{\cFunc}(\mRat)=\mRat$ up to a first `kink point' at $\mRat_{\#}^{1}>1$, and with discrete declines in the MPC at a set of kink points $\{\mRat_{\#}^{1},\mRat_{\#}^{2},...\}$.  As $\mRat \uparrow \infty$ the constrained consumption function $\mathring{\cFunc}(\mRat)$ becomes arbitrarily close to the unconstrained $\bar{\cFunc}(\mRat)$, and the marginal propensity to consume function $\mathring{\MPCFunc}(\mRat) \equiv \mathring{\cFunc}^{\prime}(\mRat)$ limits to $\MinMPC$.  Similarly, the value function $\mathring{\vFunc}(\mRat)$ is nondegenerate and limits into the value function of the unconstrained consumer.  This logic holds even when the finite human wealth condition fails (denoted \cncl{\FHWC}):  A solution exists because the constraint prevents the consumer from borrowing against infinite human wealth to finance infinite current consumption.  Under these circumstances, the consumer who starts with any amount of resources $\bRat_{t} > 1$ will, over time, run those resources down so that by some finite number of periods $n$ in the future the consumer will reach $\bRat_{t+n} = 0$, and thereafter will set $\cRat = \mRat = 1$ for eternity, a policy that will (using \eqref{eq:ValuePFAnalytical}) yield value of \hypertarget{PFFVAC}{}
\begin{align}
  \vLevBF_{t+n}  & = \uFunc(\pLevBF_{t+n})\left(1+\DiscFac
                   \PGro^{1-\CRRA}+(\DiscFac \PGro^{1-\CRRA})^{2}+...\right) \notag
  \\  & = \PGro^{n(1-\CRRA)} \uFunc(\pLevBF_{t})\left(\frac{1-(\DiscFac
        \PGro^{1-\CRRA})^{T-(t+n)+1}}{1-\DiscFac \PGro^{1-\CRRA}}\right),
        \notag
\end{align}
which will be finite whenever
\begin{align}
  \overbrace{\DiscFac \PGro^{1-\CRRA} }^{\equiv \DiscAlt}  & < 1 \nonumber
  \\ \DiscFac \Rfree \PGro^{-\CRRA}  & < \Rfree/\PGro \label{eq:PFFVAC}
  \\ \PatPGro  & < (\Rfree/\PGro)^{1/\CRRA}, \nonumber
\end{align}
where we call $\DiscAlt$ the `Perfect Foresight Value Of Autarky Factor' ({\PFFVAF})and which we call the Perfect Foresight Finite Value of Autarky Condition, \PFFVAC, because it guarantees that a consumer who always spends all permanent income will have finite value (the consumer has `finite autarky value').  Note that the version of the \PFFVAC~in \eqref{eq:PFFVAC} implies the \PFGIC~$\PatPGro < 1$ whenever \cncl{\FHWC} ($\Rfree < \PGro$) holds.  So, if \cncl{\FHWC}, value for any finite $\mRat$ will be the sum of two finite numbers: The component due to the unconstrained consumption choice made over the finite horizon leading up to $\bRat_{t+n} = 0$, and the finite component due to the value of consuming all $\pLevBF_{t+n}$ thereafter.  The consumer's value function is therefore nondegenerate.

\hypertarget{RICandFHWCFail}{}
The most peculiar possibility occurs when the \RIC~fails.  Under these circumstances the \FHWC~must also fail (Appendix~\ref{sec:ApndxLiqConstr}), and the constrained consumption function is nondegenerate.  (See appendix Figure~\ref{fig:PFGICHoldsFHWCFailsRICFails} for a numerical example).  While it is true that $\lim_{m \uparrow \infty}
\mathring{\MPCFunc}(\mRat) = 0$, nevertheless the limiting constrained
consumption function $\mathring{\cFunc}(\mRat)$ is strictly positive
and strictly increasing in $\mRat$.  This result interestingly
reconciles the conflicting intuitions from the unconstrained case,
where \cncl{\RIC} would suggest a degenerate limit of
$\mathring{\cFunc}(\mRat)=0$ while \cncl{\FHWC} would suggest a
degenerate limit of $\mathring{\cFunc}(\mRat)=\infty$.

Table~\ref{table:Comparison} and \ref{table:Required} (and appendix table~\ref{table:LiqConstrScenarios}) codify the key points to help the reader keep them straight.


An intuitive representation of the relations of the conditions is presented in Figure~\ref{fig:RelatePFGICFHWCRICPFFVAC} (for the perfect foresight case).  Each node represents a quantity considered in the foregoing analysis.  The arrow associated with each inequality condition reflects the imposition of that condition.  For example, one way of writing the {\PFFVAC} is $\Pat < \Rfree^{1/\CRRA} \PGro^{1-1/\CRRA}$, and this is captured by the diagonal arrow.  Traversing the diagram clockwise starting at $\Pat$ involves imposing first the {\PFGIC} then the {\FHWC}, and the consequent arrival at the bottom right node tells us that these two conditions jointly imply that the {\PFFVAC} holds.  Reversal of a condition will reverse the arrow's direction; so, for example, {\cncl{\FHWC}} would correspond to reversing the directon of the bottommost arrow.  This would allow us to follow the arrows in a counterclockwise direction from $\Pat$ to the bottom right node, leading to the conclusion that the combination of {\RIC} and \cncl{\FHWC} implies that the {\PFFVAC} holds.  (Readers unfamiliar with diagrams of this kind may wish to consult the expanded discussion in Appendix~\ref{sec:ApndxConditionDiagrams}).

\providecommand{\figName}{RelatePFGICFHWCRICPFFVAC} % Allows generic definition of hypertargets based on title of figure
\providecommand{\figFile}{\figName} %  and on filename
%\hypertarget{\figFile}{}
\hypertarget{\figName}{}
\input{\FigDir/\figName} % Read in the tex to generate the figure

We now turn to the case with uncertainty.  The model without constraints but with uncertainty will turn out to be a close parallel to the model with constraints but without uncertainty.

\hypertarget{Uncertainty-Modified-Conditions}{}
\subsection{Uncertainty-Modified Conditions}
\subsubsection{Impatience}

When uncertainty is introduced, the expectation of $\bRat_{t+1}$ can be rewritten as:
\begin{align}
  \Ex_{t}[\bRat_{t+1}]  & =  \aRat_{t}\Ex_{t}[\Rnorm_{t+1}] = \aRat_{t}\Rnorm\Ex_{t}[\pShk_{t+1}^{-1}] \label{eq:EbRat}
\end{align}
where Jensen's inequality guarantees that the expectation of the inverse of the permanent shock is strictly greater than one.  It will be convenient to define the object \hypertarget{InvEpShkInv}{}
\begin{align*}
  \InvEpShkInv  & \equiv  (\EpShkInv)^{-1}
\end{align*}
because this permits us to write expressions like the RHS of
\eqref{eq:EbRat} compactly as, e.g., $\aRat_{t}\Rnorm/
\InvEpShkInv^{-1}.$\footnote{One way to think of $\InvEpShkInv$ is as
  a particular kind of a `certainty equivalent' of the shock; this
  captures the intuition that mean-one shock renders a given mean
  level of income less valuable than if the shock did not exist, so
  that $\InvEpShkInv < 1$.}  We refer to this as the `compensated return,' because it compensates (in a risk-neutral way) for the effect of
uncertainty on the expected growth-normalized return (in the sense implicitly defined in
\eqref{eq:EbRat}).

\hypertarget{GIC}{}
\hypertarget{GICI}{}
We can now transparently generalize the \PFGIC~\eqref{eq:PFGIC} by defining a `compensated growth factor' \hypertarget{PGroAdj}{}
\begin{verbatimwrite}{\EqDir/PGroAdj}
  \begin{align}
    \PGroAdj  & =  \PGro \InvEpShkInv \label{eq:PGroAdj}
  \end{align}

which is equivalent to \eqref{eq:LiqConstrBinds} (exponentiate both
sides by $1/\CRRA$).

If the \RIC~\eqref{eq:RIC} and the \FHWC~\eqref{eq:FHWC} hold, appendix \ref{sec:ApndxLiqConstr} shows
that, for some $0 < \mRat_{\#} < 1$, an unconstrained consumer behaving according to
\eqref{eq:cFuncPFUnc} would choose $\cRat < \mRat$ for all $\mRat >
\mRat_{\#}$.  In this case the solution to the constrained consumer's problem is simple: For any $\mRat
\geq \mRat_{\#}$ the constraint does not bind (and will never bind in
the future) so the constrained consumption function is identical
to the unconstrained one.  If the consumer were somehow\footnote{``Somehow'' because $\mRat<1$ could only be
  obtained by entering the period with $\bRat < 0$ which the constraint
  forbids.}
to arrive at an $\mRat < \mRat_{\#} < 1$ the constraint would bind and
the consumer would have to consume $\cRat=\mRat$.  The $\circ$ accent will designate the limiting
constrained consumption function:
\begin{equation}
  \mathring{\cFunc}(\mRat)=
  \begin{cases}
    \mRat & \text{if $\mRat < \mRat_{\#}$} \\
    \bar{\cFunc}(\mRat)  & \text{if $\mRat \geq \mRat_{\#}$.}
  \end{cases}
\end{equation}

More useful is the case where the perfect foresight growth and return impatience conditions both hold.  In this case appendix \ref{sec:ApndxLiqConstr} shows that the limiting constrained consumption function is piecewise linear, with $\mathring{\cFunc}(\mRat)=\mRat$ up to a first `kink point' at $\mRat_{\#}^{1}>1$, and with discrete declines in the MPC at a set of kink points $\{\mRat_{\#}^{1},\mRat_{\#}^{2},...\}$.  As $\mRat \uparrow \infty$ the constrained consumption function $\mathring{\cFunc}(\mRat)$ becomes arbitrarily close to the unconstrained $\bar{\cFunc}(\mRat)$, and the marginal propensity to consume function $\mathring{\MPCFunc}(\mRat) \equiv \mathring{\cFunc}^{\prime}(\mRat)$ limits to $\MinMPC$.  Similarly, the value function $\mathring{\vFunc}(\mRat)$ is nondegenerate and limits into the value function of the unconstrained consumer.  This logic holds even when the finite human wealth condition fails (denoted \cncl{\FHWC}):  A solution exists because the constraint prevents the consumer from borrowing against infinite human wealth to finance infinite current consumption.  Under these circumstances, the consumer who starts with any amount of resources $\bRat_{t} > 1$ will, over time, run those resources down so that by some finite number of periods $n$ in the future the consumer will reach $\bRat_{t+n} = 0$, and thereafter will set $\cRat = \mRat = 1$ for eternity, a policy that will (using \eqref{eq:ValuePFAnalytical}) yield value of \hypertarget{PFFVAC}{}
\begin{align}
  \vLevBF_{t+n}  & = \uFunc(\pLevBF_{t+n})\left(1+\DiscFac
                   \PGro^{1-\CRRA}+(\DiscFac \PGro^{1-\CRRA})^{2}+...\right) \notag
  \\  & = \PGro^{n(1-\CRRA)} \uFunc(\pLevBF_{t})\left(\frac{1-(\DiscFac
        \PGro^{1-\CRRA})^{T-(t+n)+1}}{1-\DiscFac \PGro^{1-\CRRA}}\right),
        \notag
\end{align}
which will be finite whenever
\begin{align}
  \overbrace{\DiscFac \PGro^{1-\CRRA} }^{\equiv \DiscAlt}  & < 1 \nonumber
  \\ \DiscFac \Rfree \PGro^{-\CRRA}  & < \Rfree/\PGro \label{eq:PFFVAC}
  \\ \PatPGro  & < (\Rfree/\PGro)^{1/\CRRA}, \nonumber
\end{align}
where we call $\DiscAlt$ the `Perfect Foresight Value Of Autarky Factor' ({\PFFVAF})and which we call the Perfect Foresight Finite Value of Autarky Condition, \PFFVAC, because it guarantees that a consumer who always spends all permanent income will have finite value (the consumer has `finite autarky value').  Note that the version of the \PFFVAC~in \eqref{eq:PFFVAC} implies the \PFGIC~$\PatPGro < 1$ whenever \cncl{\FHWC} ($\Rfree < \PGro$) holds.  So, if \cncl{\FHWC}, value for any finite $\mRat$ will be the sum of two finite numbers: The component due to the unconstrained consumption choice made over the finite horizon leading up to $\bRat_{t+n} = 0$, and the finite component due to the value of consuming all $\pLevBF_{t+n}$ thereafter.  The consumer's value function is therefore nondegenerate.

\hypertarget{RICandFHWCFail}{}
The most peculiar possibility occurs when the \RIC~fails.  Under these circumstances the \FHWC~must also fail (Appendix~\ref{sec:ApndxLiqConstr}), and the constrained consumption function is nondegenerate.  (See appendix Figure~\ref{fig:PFGICHoldsFHWCFailsRICFails} for a numerical example).  While it is true that $\lim_{m \uparrow \infty}
\mathring{\MPCFunc}(\mRat) = 0$, nevertheless the limiting constrained
consumption function $\mathring{\cFunc}(\mRat)$ is strictly positive
and strictly increasing in $\mRat$.  This result interestingly
reconciles the conflicting intuitions from the unconstrained case,
where \cncl{\RIC} would suggest a degenerate limit of
$\mathring{\cFunc}(\mRat)=0$ while \cncl{\FHWC} would suggest a
degenerate limit of $\mathring{\cFunc}(\mRat)=\infty$.

Table~\ref{table:Comparison} and \ref{table:Required} (and appendix table~\ref{table:LiqConstrScenarios}) codify the key points to help the reader keep them straight.


An intuitive representation of the relations of the conditions is presented in Figure~\ref{fig:RelatePFGICFHWCRICPFFVAC} (for the perfect foresight case).  Each node represents a quantity considered in the foregoing analysis.  The arrow associated with each inequality condition reflects the imposition of that condition.  For example, one way of writing the {\PFFVAC} is $\Pat < \Rfree^{1/\CRRA} \PGro^{1-1/\CRRA}$, and this is captured by the diagonal arrow.  Traversing the diagram clockwise starting at $\Pat$ involves imposing first the {\PFGIC} then the {\FHWC}, and the consequent arrival at the bottom right node tells us that these two conditions jointly imply that the {\PFFVAC} holds.  Reversal of a condition will reverse the arrow's direction; so, for example, {\cncl{\FHWC}} would correspond to reversing the directon of the bottommost arrow.  This would allow us to follow the arrows in a counterclockwise direction from $\Pat$ to the bottom right node, leading to the conclusion that the combination of {\RIC} and \cncl{\FHWC} implies that the {\PFFVAC} holds.  (Readers unfamiliar with diagrams of this kind may wish to consult the expanded discussion in Appendix~\ref{sec:ApndxConditionDiagrams}).

\providecommand{\figName}{RelatePFGICFHWCRICPFFVAC} % Allows generic definition of hypertargets based on title of figure
\providecommand{\figFile}{\figName} %  and on filename
%\hypertarget{\figFile}{}
\hypertarget{\figName}{}
\input{\FigDir/\figName} % Read in the tex to generate the figure

We now turn to the case with uncertainty.  The model without constraints but with uncertainty will turn out to be a close parallel to the model with constraints but without uncertainty.

\hypertarget{Uncertainty-Modified-Conditions}{}
\subsection{Uncertainty-Modified Conditions}
\subsubsection{Impatience}

When uncertainty is introduced, the expectation of $\bRat_{t+1}$ can be rewritten as:
\begin{align}
  \Ex_{t}[\bRat_{t+1}]  & =  \aRat_{t}\Ex_{t}[\Rnorm_{t+1}] = \aRat_{t}\Rnorm\Ex_{t}[\pShk_{t+1}^{-1}] \label{eq:EbRat}
\end{align}
where Jensen's inequality guarantees that the expectation of the inverse of the permanent shock is strictly greater than one.  It will be convenient to define the object \hypertarget{InvEpShkInv}{}
\begin{align*}
  \InvEpShkInv  & \equiv  (\EpShkInv)^{-1}
\end{align*}
because this permits us to write expressions like the RHS of
\eqref{eq:EbRat} compactly as, e.g., $\aRat_{t}\Rnorm/
\InvEpShkInv^{-1}.$\footnote{One way to think of $\InvEpShkInv$ is as
  a particular kind of a `certainty equivalent' of the shock; this
  captures the intuition that mean-one shock renders a given mean
  level of income less valuable than if the shock did not exist, so
  that $\InvEpShkInv < 1$.}  We refer to this as the `compensated return,' because it compensates (in a risk-neutral way) for the effect of
uncertainty on the expected growth-normalized return (in the sense implicitly defined in
\eqref{eq:EbRat}).

\hypertarget{GIC}{}
\hypertarget{GICI}{}
We can now transparently generalize the \PFGIC~\eqref{eq:PFGIC} by defining a `compensated growth factor' \hypertarget{PGroAdj}{}
\begin{verbatimwrite}{\EqDir/PGroAdj}
  \begin{align}
    \PGroAdj  & =  \PGro \InvEpShkInv \label{eq:PGroAdj}
  \end{align}
